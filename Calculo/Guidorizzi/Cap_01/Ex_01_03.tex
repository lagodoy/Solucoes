\section{Módulo de um Número Real}
\begin{enumerate}
	\setcounter{enumi}{0}
	\item % S. 1.3 - Ex. 1
	\begin{enumerate}
		\item % S. 1.3 - Ex. 1 - a
		$\vert-5\vert + \vert-2\vert = -(-5) - (-2) = 5 + 2 = 7$.
		\item % S. 1.3 - Ex. 1 - b
		$\vert-5 + 8\vert = \vert 3 \vert = 3$.
		\item % S. 1.3 - Ex. 1 - c
		$\vert-a\vert = -(-a) = a$.
		\item % S. 1.3 - Ex. 1 - d
		$\vert a \vert$ , $a < 0$\\
		Como $a < 0$, $\vert a \vert = -a $
		\item % S. 1.3 - Ex. 1 - e
		$\vert -a \vert$\\
		Quando $-a \leq 0$ temos $a \geq 0$  e portanto $\vert -a \vert = -(-a)= a$.\\
		Quando $-a > 0$ temos $a < 0$ e portanto $\vert -a \vert = -a$.
		\item % S. 1.3 - Ex. 1 - f
		$\vert 2a \vert - \vert 3a \vert$\\
		$\vert 2a \vert - \vert 3a \vert = \vert 2\vert \vert a \vert - \vert 3 \vert \vert a \vert = \left( \vert 2\vert  - \vert 3 \vert \right)\vert a \vert = \left( 2  -  3 \right)\vert a \vert = \left(-1 \right)\vert a \vert$.\\
		Caso $a \geq 0$, teremos $\left(-1\right)\cdot a = -a$.\\
		Caso $a < 0$, teremos $(-1)\cdot \left(-a\right) = a$.
	\end{enumerate}
	\item % S. 1.3 - Ex. 2
	\begin{enumerate}
		\item % S. 1.3 - Ex. 2 - a
		$\vert x \vert = 2$\\
		$x = 2$ quando $x \geq 0$ ou $x = - 2$ quando $x < 0$.
		\item % S. 1.3 - Ex. 2 - b
		$\vert x + 1 \vert = 3$\\
		$x + 1 > 0 \Rightarrow x + 1 = 3 \Rightarrow x = 2$\\
		$x + 1 \leq 0 \Rightarrow -(x + 1) = 3 \Rightarrow x + 1 = -3 \Rightarrow x = -4$.
		\item % S. 1.3 - Ex. 2 - c
		$\vert 2x - 1 \vert = 1$\\
		$2x - 1 > 0 \Rightarrow 2x - 1 = 1 \Rightarrow 2x = 2 \Rightarrow x = 1$\\
		$2x - 1 \leq 0 \Rightarrow -(2x - 1) = 1 \Rightarrow 2x - 1 = -1 \Rightarrow 2x = 0 \Rightarrow x = 0$.	
		\item % S. 1.3 - Ex. 2 - d
		$\vert x - 2 \vert = -1$\\
		Não existe solução, pois o módulo de um número é sempre positivo.\\
		Podemos verificar o que ocorre quando tentamos solucionar a equação:\\
		Se $ x - 2 \geq 0$, temos $x - 2 = -1 \Rightarrow x = 1$, porém $x - 2 = 1 - 2 = -1 <  0$.\\
		Se $ x - 2  < 0$, temos $-(x - 2) = -x + 2 = -1 \Rightarrow x = 3$, porém $3 - 2 = 1 > 0$.\\
		Vemos então que existem contradições nos dois valores encontrados para $x$, logo não existe solução.
		\item % S. 1.3 - Ex. 2 - e
		$\vert2x + 3 \vert = 0$\\
		Para $2x + 3 \geq 0$, temos $2x + 3 = 0 \Rightarrow 2x = -3 \Rightarrow x = \dfrac{-3}{2}$.\\
		Para $2x + 3 < 0$, temos $-(2x + 3) = 0 \Rightarrow -2x - 3 = 0\Rightarrow -2x = 3 \Rightarrow x = \dfrac{-3}{2}$.
		\item % S. 1.3 - Ex. 2 - f
		$\vert x \vert = 2x + 1$\\
		Para $x \geq 0$, temos $x = 2x + 1 \Rightarrow x =-1$.\\
		Para $x < 0$, temos $-x = 2x + 1 \Rightarrow 3x = -1 \Rightarrow x = \dfrac{-1}{3}$.
	\end{enumerate}
	\item % S. 1.3 - Ex. 3
	\begin{enumerate}
		\setcounter{enumii}{0}
		\item % S. 1.3 - Ex. 3 - a
		$\vert x \vert \leq 1$\\
		$ x > 0 \Rightarrow x \leq 1$\\
		$ x \leq 0 \Rightarrow -x \leq 1 \Rightarrow x \geq -1$\\
		$-1 \leq x \leq 1$
		\item % S. 1.3 - Ex. 3 - b
		$\vert 2x - 1 \vert < 3$\\		
		$2x - 1 > 0 \Rightarrow 2x - 1 < 3 \Rightarrow 2x < 4 \Rightarrow x < 2$\\
		$2x - 1 < 0 \Rightarrow -(2x - 1) < 3 \Rightarrow 2x - 1 > -3  \Rightarrow 2x > -2 \Rightarrow x > -1$\\
		$-1 < x < 2$
		\item % S. 1.3 - Ex. 3 - c
		$\vert 2x - 1 \vert < -2$, não admite solução pois o módulo de um número real é sempre positivo ou igual à 0.
		\item % S. 1.3 - Ex. 3 - d
		$\vert 2x - 1 \vert < \frac{1}{3}$\\
		$-\frac{1}{3} < 2x - 1 < \frac{1}{3} \Rightarrow -\frac{1}{3} + 1 < 3x < \frac{1}{3} + 1 \Rightarrow$\\
		$\frac{2}{3} < 3x < \frac{4}{3} \Rightarrow \frac{2}{9} < x < \frac{4}{9}$.
		\item % S. 1.3 - Ex. 3 - e
		$\vert 2x^2 - 1 \vert < 1$\\
		$\vert 2x^2 - 1 \vert > 0 \Rightarrow 2x^2 - 1 < 1 \Rightarrow 2x^2 < 2 \Rightarrow x^2 < 1 \Rightarrow x < 1$ ou $x > -1$ com $x \neq 0$.
		$\vert 2x^2 - 1 \vert \leq 0 \Rightarrow 2x^2 - 1 > -1 \Rightarrow 2x^2 > 0 \Rightarrow x^2 > 0 \Rightarrow x \neq 0$\\
		$-1 < x < 1$, $x \neq 0$.
		\item % S. 1.3 - Ex. 3 - f
		$\vert x - 3 \vert < 4$\\
		$-4 <  x - 3 < 4 \Rightarrow -1 < x < 7$.
		\item % S. 1.3 - Ex. 3 - g
		$\vert x \vert > 3$\\		
		$x > 0 \Rightarrow x > 3$\\
		$x \leq 0 \Rightarrow -x > 3 \Rightarrow x < -3$\\
		$x < -3$ ou $x > 3$.
		\item % S. 1.3 - Ex. 3 - h
		$\vert x + 3 \vert > 1$\\
		$\vert x + 3 \vert > 1 \Leftrightarrow \vert x + 3\vert^2 > 1^2  \Leftrightarrow (x+3)^2 > 1^2 \Leftrightarrow (x+3)^2 - 1^2 > 0 \Leftrightarrow$\\
		$\Leftrightarrow [(x+3)-1][(x+3)+1]>0 \Leftrightarrow (x+2)(x+4)>0$\\
		Para se obter $(x+2)(x+4)>0$, devemos ter as expressões $(x+2)$ e $(x+4)$ com mesmo sinal:\\
		$x+2 < 0$ e $ x+4 < 0 $ nos dá $x < -2$ e $x < -4$, logo devemos ter $x < -4$\\
		$x+2 > 0$ e $ x+4 > 0 $ nos dá $x > -2$ e $x > -4$, logo devemos ter $x > -2$\\
		A solução é $x < -4$ ou $x > -2$
		\item % S. 1.3 - Ex. 3 - i
		$\vert 2x - 3 \vert > 3$\\
		$\vert 2x - 3 \vert > 3 \Leftrightarrow \vert 2x - 3 \vert^2 > 3^2 \Leftrightarrow (2x - 3)^2 > 3^2 \Leftrightarrow $\\
		$\Leftrightarrow (2x - 3)^2 - 3^2 > 0 \Leftrightarrow (2x - 3 - 3)(2x - 3 + 3) >  0 \Leftrightarrow (2x-6)(2x) > 0$ \\
		Para se obter  $(2x-6)(2x) > 0$, devemos ter as expressões  $(2x-6)$ e $2x$  com mesmo sinal:\\
		$2x-6 < 0$ e $2x < 0$,  resulta em $x < 3$ e $x < 0$, logo deve-se ter $x < 0$.\\
		$2x-6 > 0$ e $2x > 0$,  resulta em $x > 3$ e $x > 0$, logo deve-se ter $x > 3$.
		\item % S. 1.3 - Ex. 3 - j
		$\vert 2x - 1 \vert < x$\\
		$\vert 2x - 1 \vert < x \Leftrightarrow \vert 2x - 1 \vert^2 < x^2 \Leftrightarrow ( 2x - 1 )^2 < x^2 \Leftrightarrow$\\
		$\Leftrightarrow (2x-1)^2 - x^2 < 0 \Leftrightarrow (2x - 1 -x)(2x-1+x) < 0 \Leftrightarrow$\\
		$\Leftrightarrow (x-1)(3x-1) < 0$
		Para se obter  $(x-1)(3x-1) < 0$,as expressões $(x-1)$ e $(3x-1)$ devem ter sinais opostos:\\
		$x-1 >0$ e $3x-1 <0$, resulta em $x > 1$ e $x < 1/3$, que não soluciona a inequação.\\
		$x-1 <0$ e $3x-1 >0$, resulta em $x < 1$ e $x > 1/3$, que resulta no intervalo $\dfrac{1}{3}<x<1$.
		\addtocounter{enumii}{1}
		\item % S. 1.3 - Ex. 3 - l
		$\vert x+ 1 \vert < \vert 2x - 1 \vert$\\
		$\vert x+ 1 \vert < \vert 2x - 1 \vert \Leftrightarrow \vert x+ 1 \vert^2 < \vert 2x - 1 \vert^2 \Leftrightarrow ( x + 1 )^2 < ( 2x - 1 )^2 \Leftrightarrow$\\ 
		$\Leftrightarrow ( x + 1 )^2 - ( 2x - 1 )^2 < 0 \Leftrightarrow$\\
		$\Leftrightarrow [( x + 1 ) - ( 2x - 1 )]\cdot[( x + 1 ) - ( 2x - 1 )] < 0 \Leftrightarrow$\\ 
		$\Leftrightarrow (x + 1 -2x +1)(x+1+2x-1) \Leftrightarrow (-x+2)(3x)<0$.\\
		Para se obter  $(-x+2)(3x)<0$,as expressões $(-x+2)$ e $3x$ devem ter sinais opostos:\\
		$-x+2 < 0$ e $3x > 0$,  resulta em $x > 2$ e $x > 0$, logo deve-se ter $x > 2$.\\
		$-x+2 > 0$ e $3x < 0$,  resulta em $x < 2$ e $x < 0$, logo deve-se ter $x < 0$.
		\item % S. 1.3 - Ex. 3 - m
		$\vert x - 1 \vert - \vert x + 2 \vert > x$\\
		Neste caso é necessário avaliar quatro combinações com relação aos resultados dos módulos, de acordo com o sinal da expressão no módulo:
		\begin{itemize}
		\item
		$x-1 > 0$ e $x +2 >0$, resulta em $x > 1$ e $x > -2$, logo tem-se essa combinação com $x > 1$\\
		$x-1-x+2>x \Leftrightarrow 1>x \Leftrightarrow x <1$
		\item
		$x-1 > 0$ e $x +2 <0$, resulta em $x > 1$ e $x < -2$, que não é possível.
		\item
		$x-1 < 0$ e $x +2 >0$, resulta em $x < 1$ e $x > -2$, logo tem-se essa combinação com $-2 < x < 1$\\
		$-(x-1) -(x + 2) >x \Leftrightarrow -x+1 - x -2 >x \Leftrightarrow  -2x-1>x \Leftrightarrow x -1 > 3x \Leftrightarrow x < \dfrac{-1}{3}$
		\item
		$x-1 < 0$ e $x +2 <0$, resulta em $x < 1$ e $x < -2$, logo tem-se essa combinação com $x < -2$\\
		$-(x-1) -[-(x+2)] >x \Leftrightarrow -x + 1 +x + 2 > x \Leftrightarrow 3 >x \Leftrightarrow x < 3$.
		\end{itemize}
		Finalmente, dos resultados acima, chega-se ao resultado $x < \dfrac{-1}{3}$
		\item % S. 1.3 - Ex. 3 - n
		$\vert x - 3 \vert < x + 1$\\
		$\vert x - 3 \vert < x + 1 \Leftrightarrow \vert x - 3 \vert^2 < (x + 1)^2 \Leftrightarrow (x - 3)^2 < (x + 1)^2 \Leftrightarrow$\\
		$\Leftrightarrow (x - 3)^2 - (x + 1)^2 < 0 \Leftrightarrow [(x - 3) - (x + 1)]\cdot[(x - 3) + (x + 1)] < 0\Leftrightarrow$\\
		$\Leftrightarrow (x - 3 - x - 1)(x - 3 + x + 1) < 0 \Leftrightarrow -4(2x - 2) < 0 \Leftrightarrow -(2x - 2) < 0 \Leftrightarrow$\\
		$\Leftrightarrow -2x + 2 < 0 \Leftrightarrow -2x < - 2 \Leftrightarrow x > 1$.
		\item % S. 1.3 - Ex. 3 - o
		$\vert x - 2 \vert + \vert x-1 \vert > 1$\\
		Quando $x - 2 < 0 \Rightarrow x < 2$, $\vert x - 2 \vert = -x+2$. Já caso $x - 2 \geq 0 \Rightarrow x \geq 2$,  $\vert x - 2 \vert = x - 2$.\\ 
		 Quando $x-1<0 \Rightarrow x < 1$, $\vert x-1 \vert = -x + 1$. Já caso $x-1 \geq 0 \Rightarrow x \geq 1$, $\vert x - 1 \vert = x - 1$.
		 \begin{itemize}
		 \item 
		 Quando $x -2 < 0$ e $x - 1 < 0$, $x < 1$, tem-se:\\
		 $-x + 2 - x - 1 > 1 \Leftrightarrow -2x +1 > 1 \Leftrightarrow x < 0$
		 \item
		 Quando $x - 2 < 0$ e $x -1 \geq 0$, $1 \leq x < 2$.\\
		 $-x + 2 + x -1 = 1 > 1$. Sem solução nesse caso.
		 \item
		 Quando $x - 2 \geq 0$ e $x - 1 \geq 0$, $x \geq 2$, tem-se:\\
		 $x-2+x-1>1 \Leftrightarrow 2x-3 > 1 \Leftrightarrow 2x > 4 \Leftrightarrow x > 2$.
		 \end{itemize}
 		 A solução da inequação é $x<1$ ou  $x>2$.
	\end{enumerate}
	\item 
		Dado $r > 0$, provar: \\
		$\vert x \vert > r \Leftrightarrow x < -r$ ou $x > r$\\
		$x > 0 \Rightarrow x > r$\\
		$x \leq 0 \Rightarrow -x > r \Rightarrow x < -r$\\
		Logo $\vert x \vert > r \Rightarrow x < -r$ ou $x > r$.\\
		Por outro lado:\\
		$x > r$ com $r > 0 \Rightarrow x^2 > r^2 \Rightarrow \sqrt{x^2} > \sqrt{r^2} \Rightarrow \vert x \vert > r$.\\
		No caso de $x < -r$ com $r > 0$ temos:\\
		$x < -r$ com $x < 0 \Rightarrow -x > r \Rightarrow (-x)^2 > r^2\Rightarrow \sqrt{(-x)^2} > \sqrt{r^2} \Rightarrow \vert x \vert > r$.
	\item % S. 1.3 - Ex. 5
		\begin{enumerate}
			\item % S. 1.3 - Ex. 5 - a
			$\vert x + 1 \vert + \vert x \vert$\\
			Devemos averiguar as quatro combinações de sinais para as duas expressões nos módulos:\\
			Para $x + 1 > 0$ e $x > 0$, temos $x > -1$ e $x > 0$, ou seja, $x > 0$:\\
			$x + 1 + x = 2x + 1$\\
			Para $x + 1 > 0$ e $x \leq 0$, temos $x > -1$ e $x \leq 0$, ou seja, $-1 < x \leq 0$:\\
			$x + 1 - x = 1$\\
			Para $x + 1 \leq 0$ e $x > 0$, temos $x < -1$ e $x > 0$, que não é possível.\\
			Para $x + 1 \leq 0$ e $x \leq 0$, temos $x \leq -1$ e $x \leq 0$, ou seja, $x \leq -1$:\\		
			$-(x + 1) -x = -2x - 1$\\
			Logo a solução é:\\
			\begin{equation*}
		    	\vert x + 1 \vert + \vert x \vert =
			    \begin{cases}
			      -2x - 1, & \text{se}\ x \leq -1 \\
			      \hfill 1, & \text{se}\ -1 < x \leq 0 \\
			      \hfill 2x + 1, & \text{se}\ x > 0 
		    	\end{cases}
			\end{equation*}
			\item % S. 1.3 - Ex. 5 - b
			$\vert x - 2\vert - \vert x + 1 \vert$\\
			Para $x - 2 > 0$ e $x + 1 > 0$, temos $x > 2$ e $x > -1$, ou seja, $x > 2$:\\
			$x - 2 - x - 1 = -3$\\
			Para $x - 2 > 0$ e $x + 1 \leq 0$, não é possível haver $x > 2$ e $ x < -1$.\\
			Para $x - 2 \leq 0$ e $x + 1 > 0$, temos $x \leq 2$ e $x > -1$, ou seja, $-1 < x \leq 2$:\\
			$-(x - 2) - (x + 1) = -x + 2 - x - 1 = -2x + 1$\\
			Para $x - 2 \leq 0$ e $x + 1 \leq 0$, temos $x \leq 2$ e $x \leq -1$, ou seja, $x \leq -1$:\\
			$-(x - 2) - [-(x + 1)] = -x + 2 + x + 1 = 3$\\
			Logo a solução é:\\
			\begin{equation*}
		    	\vert x - 2 \vert - \vert x + 1 \vert =
			    \begin{cases}
			      \hfill 3, & \text{se}\ x \leq -1 \\
			      -2x + 1, & \text{se}\ -1 < x \leq 2 \\
			      \hfill -3, & \text{se}\ x > 2 
		    	\end{cases}
			\end{equation*}
			\item % S. 1.3 - Ex. 5 - c
			$\vert 2x -1 \vert + \vert x -2\vert$\\
			Para $x\leq\dfrac{1}{2}$, $2x - 1 < 0$ e $x - 2 <0$, assim temos:\\
			$-(2x-1)-(x-2) = -2x+1-x+2=-3x+3$.\\
			Para $x \geq 2$, $2x - 1 \geq 0$ e $x - 2 \geq 0$, assim temos:\\
			$2x-1+x-2=3x-3$.\\
			Para $\dfrac{1}{2} \leq x < 2$, $2x - 1 \geq 0$ e $x - 2 <0$, assim temos:\\
			$2x-1-(x-2)=2x-1-x+2=x+1$.\\
			Logo a solução é:\\
			\begin{equation*}
		    	\vert 2x -1 \vert + \vert x -2\vert =
			    \begin{cases}
			      \hfill -3x+3, & \text{se}\ x\leq\dfrac{1}{2} \\
			      \hfill x+1, & \text{se}\ \dfrac{1}{2} < x < 2\\
			      \hfill 3x-3, & \text{se}\ x \geq 2 
		    	\end{cases}
			\end{equation*}
			\item % S. 1.3 - Ex. 5 - d
			$\vert x \vert + \vert x - 1\vert + \vert x - 2 \vert$\\
			Para $x \leq 0$, $x \leq 0$, $x - 1 < 0$ e $x -2 <0$, assim temos:\\
			$-(x)-(x-1)-(x-2)=-x-x+1-x+2=-3x+3$.\\
			Para $x \geq 2$, $x > 0$, $x-1 >0$ e $x -2 \leq 0$, assim temos:\\
			$(x)+(x-1)+(x-2)=3x-3$.\\
			Para $0 < x \leq 1$, $x>0$, $x -1 \leq 0$ e $x - 2 < 0$, assim temos:\\
			$(x) -(x-1)-(x-2)=x-x+1-x+2=-x+3$.\\
			Para $1 < x \geq 2$, $x>0$, $x -1 > 0$ e $x - 2 \leq 0$, assim temos:\\
			$(x) + (x -1)-(x-2)=x+x-1-x+2=x+1$.\\
			Logo a solução é:\\
			\begin{equation*}
		    	\vert x \vert + \vert x - 1\vert + \vert x - 2 \vert =
			    \begin{cases}
			      \hfill -3x+3, & \text{se}\ x \leq 0 \\
			      \hfil -x+3, & \text{se}\ 0 < x \leq 1 \\
			      \hfill x+1, & \text{se}\ 1 < x \leq 2\\
			      \hfill 3x-3, & \text{se}\ x \geq 2
		    	\end{cases}
			\end{equation*}
		\end{enumerate}
		\item % S. 1.3 - Ex. 6
		$\vert x + y \vert = \vert x \vert + \vert y \vert \Leftrightarrow \left(\vert x + y \vert\right)^2 = \left(\vert x \vert + \vert y \vert\right)^2 \Leftrightarrow$\\
		$\Leftrightarrow (x+y)^2 = \vert x \vert^2+2\vert x \vert\cdot\vert y \vert+ \vert y \vert^2 \Leftrightarrow $\\
		$\Leftrightarrow x^2+2xy+y^2 = x^2 +2\vert x \vert\cdot\vert y \vert+ y^2 \Leftrightarrow $\\
		$\Leftrightarrow xy = \vert x \vert\cdot\vert y \vert \Leftrightarrow xy= \vert xy \vert$\\
		Pela definição do módulo de um número real, $\vert xy \vert \geq 0$, logo $xy \geq 0$.\\
		Assim conclui-se: $\vert x + y \vert = \vert x \vert + \vert y \vert \Leftrightarrow xy= \vert xy \vert \Leftrightarrow xy \geq 0$
		\item % S. 1.3 - Ex. 7
			\begin{enumerate}
				\item % S. 1.3 - Ex. 7 - a
				$\vert x - y \vert \geq \vert x\vert - \vert y\vert$\\
				$\vert x \vert = \vert x + y - y \vert \leq \vert x - y \vert + \vert y \vert \Leftrightarrow \vert x - y \vert \geq \vert x \vert - \vert y \vert$
				\item % S. 1.3 - Ex. 7 - b
				$\vert x - y \vert \geq \vert y\vert - \vert x\vert$\\
				$\vert y \vert = \vert y - x + x \vert  = \vert -(x - y) + x \vert \leq \vert -(x - y) \vert + \vert x \vert = \vert  x - y \vert + \vert x \vert \Leftrightarrow \vert x - y \vert \geq \vert y \vert - \vert x \vert $
				\item % S. 1.3 - Ex. 7 - c
				$\vert \vert x \vert - \vert y \vert \vert \leq \vert x - y\vert$\\
				$\vert x \vert  - \vert y \vert \geq 0$ resulta em  $ \vert \vert x \vert  - \vert y \vert \vert =\vert x \vert  - \vert y \vert \leq  \vert x-y \vert$ e comforme item a) acima fica provada a desigualdade.\\
				$\vert x \vert  - \vert y \vert \leq 0$ resulta em  $ \vert \vert x \vert  - \vert y \vert \vert =\vert y \vert  - \vert x \vert \leq \vert x - y  \vert$ e comforme item b) acima fica provada a desigualdade.\\
					Logo $\vert \vert x \vert - \vert y \vert \vert \leq \vert x - y\vert$.
			\end{enumerate}
\end{enumerate}