\section{Funções de Uma Variável Real a Valores Reais}
\begin{enumerate}
\item % S. 2.1 - Ex. 1
	\begin{enumerate}
		\addtocounter{enumii}{2}
		\item % S. 2.1 - Ex. 1 - c
			$f(x)=x^2$ e $ab \neq 0$.
		\begin{equation*}
		\begin{split}
			\dfrac{f(a+b)-f(a-b)}{ab} &= \dfrac{(a+b)^2-(a-b)^2}{ab} = \dfrac{a^2+2ab+b^2-(a^2-2ab+b^2)}{ab} = \\ 
			&=\dfrac{a^2+2ab+b^2-a^2+2ab-b^2}{ab}=\dfrac{4ab}{ab}=4
		\end{split}
		\end{equation*}
		\item % S. 2.1 - Ex. 1 - d
		$f(x)=3x+1$ e $ab\neq 0$.\\
		\begin{equation*}
		\begin{split}
		\dfrac{f(a+b)-f(a-b)}{ab}&= \dfrac{3(a+b)+1-[3(a-b)+1]}{ab}=\dfrac{3a+3b+1-(3a-3b+1)}{ab}=\\
								  &= \dfrac{3a+3b+1-3a+3b-1}{ab}=\\
								  &= \dfrac{6b}{ab}=\dfrac{6}{a}
		\end{split}
		\end{equation*}
		
	\end{enumerate}
\item % S. 2.1 - Ex. 2
	Simplifique $\dfrac{f(x)-f(p)}{x-p}$ ($x\neq p$):
	\begin{enumerate}
		\addtocounter{enumii}{8}
		\item % S. 2.1 - Ex. 2 - i
		$f(x)=x^3$ e $p$ qualquer.
		\begin{equation*}
		\dfrac{f(x)-f(p)}{x-p}=\dfrac{x^3-p^3}{x-p}=\dfrac{(x-p)(x^2+xp+p^2)}{x-p}=x^2+px+p^2		
		\end{equation*}
		\item % S. 2.1 - Ex. 2 - j
		$f(x)=\dfrac{1}{x}$ e $p=1$
		\begin{equation*}
		\dfrac{f(x)-f(p)}{x-p} = \dfrac{\dfrac{1}{x}-\dfrac{1}{1}}{x-1	}=\dfrac{\dfrac{1-x}{x}}{\dfrac{x-1}{1}}=\dfrac{1-x}{x(x-1)}=-\dfrac{x-1}{x(x-1)}=-\dfrac{1}{x}
		\end{equation*}
		\addtocounter{enumii}{1}
		\item % S. 2.1 - Ex. 2 - l
		$f(x)=\dfrac{1}{x}$ e $p=2$
		\begin{equation*}		
		\dfrac{f(x)-f(p)}{x-p} = \dfrac{\dfrac{1}{x}-\dfrac{1}{2}}{x-2}=\dfrac{\dfrac{2-x}{2x}}{x-2}=\dfrac{2-x}{2x(x-2)}=-\dfrac{x-2}{2x(x-2)}=-\dfrac{1}{2x}
		\end{equation*}
		\item % S. 2.1 - Ex. 2 - m
		$f(x)=x^2-3x$ e $p=-2$
		\begin{equation*}
		\begin{split}
		\dfrac{f(x)-f(p)}{x-p}&= \dfrac{x^2-3x-[(-2)^2-3\cdot(-2)]}{x-(-2)}=\dfrac{x^2-3x-(4+6)}{x+2}\\
		 					   &= \dfrac{x^2-3x-10}{x+2}=\dfrac{(x+2)(x-5)}{x+2}=x-5
		\end{split}
		\end{equation*}
		\item % S. 2.1 - Ex. 2 - n
		$f(x)=\dfrac{1}{x^2}$ e $p=3$
		\begin{equation*}
		\dfrac{f(x)-f(p)}{x-p} = \dfrac{\dfrac{1}{x^2}-\dfrac{1}{3^2}}{x-3}=\dfrac{\dfrac{3^2-x^2}{x^2\cdot3^2}}{x-3}=\dfrac{3^2-x^2}{3^2x^2(x-3)}=-\dfrac{x^2-3^2}{3^2x^2(x-3)}=-\dfrac{(x-3)(x+3)}{3^2x^2(x-3)}=-\dfrac{x+3}{9x^2}
		\end{equation*}
		\item % S. 2.1 - Ex. 2 - o
		$f(x)=\dfrac{1}{x^2}$ e $p=-3$
		\begin{equation*}
		\begin{split}
		\dfrac{f(x)-f(p)}{x-p}&=\dfrac{\dfrac{1}{x^2}-\dfrac{1}{(-3)^2}}{x-(-3)}=\dfrac{\dfrac{(-3)^2-x^2}{(-3)^2x^2}}{x+3}=\dfrac{(-3)^2-x^2}{(-3)^2x^2(x+3)}=-\dfrac{x^2-(-3)^2}{(-3)^2x^2(x+3)}=\\
							  &=-\dfrac{(x+3)(x-3)}{(-3)^2x^2(x+3)}=-\dfrac{x-3}{9x^2}
		\end{split}
		\end{equation*}
		\item % S. 2.1 - Ex. 2 - p
		$f(x)=\dfrac{1}{x}$ e $p\neq0$
		\begin{equation*}
		\dfrac{f(x)-f(p)}{x-p}=\dfrac{\dfrac{1}{x}-\dfrac{1}{p}}{x-p}=\dfrac{\dfrac{p-x}{xp}}{x-p}=\dfrac{p-x}{px(x-p)}=-\dfrac{x-p}{px(x-p)}=-\dfrac{1}{px}
		\end{equation*}
		\item % S. 2.1 - Ex. 2 - q
		$f(x)=\dfrac{1}{x^2}$ e $p\neq0$
		\begin{equation*}
		\dfrac{f(x)-f(p)}{x-p}=\dfrac{\dfrac{1}{x^2}-\dfrac{1}{p^2}}{x-p}=\dfrac{\dfrac{p^2-x^2}{x^2p^2}}{x-p}=\dfrac{p^2-x^2}{p^2x^2(x-p)}=-\dfrac{x^2-p^2}{p^2x^2(x-p)}=-\dfrac{(x-p)(x+p)}{p^2x^2(x-p)}=-\dfrac{x+p}{p^2x^2}
		\end{equation*}
	\end{enumerate}
\item % S. 2.1 - Ex. 3
	Simplifique $\dfrac{f(x+h)-f(x)}{h}$ ($h\neq 0$):
	\begin{enumerate}
		\addtocounter{enumii}{9}
		\item % S. 2.1 - Ex. 3 - j
		$f(x)=2x^2+x+1$
		\begin{equation*}
		\begin{split}
		\dfrac{f(x+h)-f(x)}{h}&=\dfrac{2(x+h)^2+(x+h)+1-(2x^2+x+1)}{h}=\\
							&= \dfrac{2(x^2+2xh+h^2)+x+h+1-2x^2-x-1}{h}\\
							&=\dfrac{2x^2+4xh+2h^2+h-2x^2}{h}\\
							&=\dfrac{4xh+2h^2+h}{h}=4x+2h+1
		\end{split}
		\end{equation*}
		\addtocounter{enumii}{1}
		\item % S. 2.1 - Ex. 3 - l
		$f(x)=x^3$
		\begin{equation*}
		\begin{split}
		\dfrac{f(x+h)-f(x)}{h}&=\dfrac{(x+h)^3-x^3}{h}=\\
							   &=\dfrac{x^3+3x^2h+3xh^2+h^3-x^3}{h}\\
							   &=\dfrac{3x^2h+3xh^2+h^3}{h}=3x^2+3xh+h^2
		\end{split}
		\end{equation*}		
		\item % S. 2.1 - Ex. 3 - m
		$f(x)=x^3+2x$
		\begin{equation*}
		\begin{split}
		\dfrac{f(x+h)-f(x)}{h}&=\dfrac{(x+h)^3+2(x+h)-(x^3+2x)}{h}=\\
							  &=\dfrac{x^3+3x^2h+3xh^2+h^3+2x+2h-x^3-2x}{h}=\\
							  &=\dfrac{3x^2h+3xh^2+h^3+2h}{h}=3x^2+3xh+h^2+2
		\end{split}
		\end{equation*}		
		\item % S. 2.1 - Ex. 3 - n
		$f(x)=x^3+x^2-x$
		\begin{equation*}
		\begin{split}
		\dfrac{f(x+h)-f(x)}{h}&=\dfrac{(x+h)^3+(x+h)^2-(x+h)-(x^3+x^2-x)}{h}=\\
							   &=\dfrac{x^3+3x^2h+3xh^2+h^3+x^2+2xh+h^2-x-h-x^3-x^2-x}{h}=\\
							   &=\dfrac{3x^2h+3xh^2+h^3+2hx+h^2-h}{h}=\\
							   &=3x^2+3xh+h^2+2x+h-1
		\end{split}
		\end{equation*}		
		\item % S. 2.1 - Ex. 3 - o
		$f(x)=5$
		\begin{equation*}
		\dfrac{f(x+h)-f(x)}{h}=\dfrac{5-5}{h}=0
		\end{equation*}		
		\item % S. 2.1 - Ex. 3 - p
		$f(x)=\dfrac{1}{x}$
		\begin{equation*}
		\begin{split}
		\dfrac{f(x+h)-f(x)}{h}=\dfrac{\dfrac{1}{x+h}-\dfrac{1}{x}}{h}=\dfrac{\dfrac{x-(x+h)}{x(x+h)}}{h}=\dfrac{x-x-h}{x(x+h)h}=-\dfrac{h}{x(x+h)h}=-\dfrac{1}{x(x+h)}
		\end{split}
		\end{equation*}
		\item % S. 2.1 - Ex. 3 - q
		$f(x)=2x^3-x$
		\begin{equation*}
		\begin{split}
		\dfrac{f(x+h)-f(x)}{h}&=\dfrac{2(x+h)^3-(x+h)-(2x^3-x)}{h}=\\
								&=\dfrac{2(x^3+3x^2h+3xh^2+h^3)-x-h-2x^3+x}{h}=\\
								&=\dfrac{2x^3+6x^2h+6xh^2+2h^3-x-h-2x^3+x}{h}=\\
								&=\dfrac{6x^2h+6xh^2+2h^3-h}{h}=6x^2+6xh+2h^2-1
		\end{split}
		\end{equation*}		
		\item % S. 2.1 - Ex. 3 - r
		$f(x)=\dfrac{1}{x^2}$
		\begin{equation*}
		\begin{split}
		\dfrac{f(x+h)-f(x)}{h}&=\dfrac{\dfrac{1}{(x+h)^2}-\dfrac{1}{x^2}}{h}=\dfrac{\dfrac{x^2-(x+h)^2}{(x+h)^2x^2}}{h}=\dfrac{\dfrac{x^2-(x^2+2xh+h^2)}{(x+h)^2x^2}}{h}=\\
								&=\dfrac{x^2-x^2-2xh-h^2}{(x+h)^2x^2h}=\dfrac{-2xh-h^2}{(x+h)^2x^2h}=-\dfrac{2x+h}{(x+h)^2x^2}
		\end{split}
		\end{equation*}		
		\item % S. 2.1 - Ex. 3 - s
		$f(x)=\dfrac{1}{x+2}$
		\begin{equation*}
		\begin{split}
		\dfrac{f(x+h)-f(x)}{h}&=\dfrac{\dfrac{1}{x+h+2}-\dfrac{1}{x+2}}{h}=\dfrac{\dfrac{x+2-(x+h+2)}{(x+h+2)(x+2)}}{h}=\dfrac{x+2-x-h-2}{(x+h+2)(x+2)h}=\\
								&=-\dfrac{h}{(x+h+2)(x+2)h}=-\dfrac{1}{(x+h+2)(x+2)}
		\end{split}
		\end{equation*}		
	\end{enumerate}
\item % S. 2.1 - Ex. 4
	\begin{enumerate}
		\addtocounter{enumii}{9}
		\item % S. 2.1 - Ex. 4 - j
		\begin{equation*}
		    	g(x) =
			    \begin{cases}
			      \hfill x, & \text{se}\ x \leq 2 \\
			      \hfill 3, & \text{se}\ x > 2 \\
		    	\end{cases}
		\end{equation*}
		$D_g = \mathbb{R}$\\
    	
		\begin{tikzpicture}[>=latex]
			\begin{axis}[
		  		axis x line=center,
		  		axis y line=center,
		  		%xtick={-4,...,4},
		  		xtick={2},
		  		%ytick={-4,...,4},
		  		ytick={2,3},
		  		%yticklabels={-A$\flat$, -A, -B$\flat$, -B, -C, C, C$\sharp$, D, D$\sharp$, E},
			  	xlabel={$x$},
			  	ylabel={$y$},
			  	xlabel style={below right},
			  	ylabel style={above left},
			  	xmin=-4.5,
			  	xmax=4.5,
				ymin=-4.5,
		  		ymax=4.5]
			\addplot [mark=none,domain=-3:2] {x};		
			\addplot [mark=none,domain=2.1:4] {3};
			\end{axis}
			%\draw[clip] (650,750) circle (0.07cm);
			\draw[dashed] (4.9525,2.8) -- (4.9525,4.1115) -- (3.35,4.1115);
			\draw[dashed] (4.9525,4.115) -- (4.9525,4.615);
			\fill (4.95,4.1) circle (0.07cm);
			\draw[clip] (4.95,4.74) circle (0.07cm);
		\end{tikzpicture}
		
	\addtocounter{enumii}{1}
	
	\item % S. 2.1 - Ex. 4 - l
		\begin{equation*}
		    	f(x) =
			    \begin{cases}
			      \hfill 2x, & \text{se}\ x \leq -1 \\
			      \hfill -x+1, & \text{se}\ x > -1 \\
		    	\end{cases}
		\end{equation*}
		$D_f = \mathbb{R}$\\
		\begin{tikzpicture}[>=latex]
			\begin{axis}[
		  		axis x line=center,
		  		axis y line=center,
		  		%xtick={-4,...,4},
		  		xtick={-1},
		  		xticklabel style={below left},
		  		%ytick={-4,...,4},
		  		ytick={-2,2},
			  	yticklabel style={right},
			  	yticklabels={\phantom{2}-2,\phantom{2}2},
		  		%yticklabels={-A$\flat$, -A, -B$\flat$, -B, -C, C, C$\sharp$, D, D$\sharp$, E},
			  	xlabel={$x$},
			  	ylabel={$y$},
			  	xlabel style={below right},
			  	ylabel style={above left},
			  	xmin=-4.5,
			  	xmax=4.5,
				ymin=-4.5,
		  		ymax=4.5]
			\addplot [mark=none,domain=-3:-1] {2*x};		
			\addplot [mark=none,domain=-0.94:3] {-1*x+1};
			\end{axis}
			%\draw[clip] (650,750) circle (0.07cm);
			\draw[dashed] (3.45,4.1145) -- (2.7555,4.1145);% -- (2.6645,1.581) -- (3.55,1.581);
			\draw[dashed] (2.6645,4.0145) -- (2.6645,1.581) -- (3.45,1.581);
			\fill (2.6645,1.581) circle (0.07cm);
			\draw[clip] (2.6645,4.1145) circle (0.07cm);
		\end{tikzpicture}		
	\item % S. 2.1 - Ex. 4 - m
		\begin{equation*}
		    	h(x) = \vert x - 1 \vert
		\end{equation*}	
		$D_h = \mathbb{R}$\\
		\begin{tikzpicture}[>=latex]
			\begin{axis}[
		  		axis x line=center,
		  		axis y line=center,
		  		%xtick={-4,...,4},
		  		xtick={1,2},
		  		%ytick={-4,...,4},
		  		ytick={1},
			  	ylabel style={above left},
		  		%yticklabels={-A$\flat$, -A, -B$\flat$, -B, -C, C, C$\sharp$, D, D$\sharp$, E},
			  	xlabel={$x$},
			  	ylabel={$y$},
			  	xlabel style={below right},
			  	ylabel style={above left},
			  	xmin=-4.5,
			  	xmax=4.5,
				ymin=-4.5,
		  		ymax=4.5]
			\addplot [mark=none,domain=-2:4] {abs(x-1)};		
			%\addplot [mark=none,domain=-0.94:3] {-1*x+1};
			\end{axis}
			%\draw[clip] (650,750) circle (0.07cm);
			\draw[dashed] (3.45,3.478) -- (4.9525,3.478) -- (4.9525,2.8) ; %-- (3.55,1.581);
			%\fill (2.65,1.55) circle (0.07cm);
			%\draw[clip] (2.6645,4.1145) circle (0.07cm);
		\end{tikzpicture}		
	\item % S. 2.1 - Ex. 4 - n
		\begin{equation*}
		    	f(x) = \vert x + 2 \vert
		\end{equation*}	
		$D_f = \mathbb{R}$\\
		\begin{tikzpicture}[>=latex]
			\begin{axis}[
		  		axis x line=center,
		  		axis y line=center,
		  		%xtick={-4,...,4},
		  		xtick={-4,-2},
		  		%ytick={-4,...,4},
		  		ytick={2},
		  		yticklabel style={right},
			  	yticklabels={\phantom{1}2},
		  		%yticklabels={-A$\flat$, -A, -B$\flat$, -B, -C, C, C$\sharp$, D, D$\sharp$, E},
			  	xlabel={$x$},
			  	ylabel={$y$},
			  	xlabel style={below right},
			  	ylabel style={above left},
			  	xmin=-4.5,
			  	xmax=4.5,
				ymin=-4.5,
		  		ymax=4.5]
			\addplot [mark=none,domain=-5:1] {abs(x+2)};		
			%\addplot [mark=none,domain=-0.94:3] {-1*x+1};
			\end{axis}
			%\draw[clip] (650,750) circle (0.07cm);
			\draw[dashed] (0.382,2.8) -- (0.382,4.11) -- (3.45,4.11) ; %-- (3.55,1.581);
			%\fill (2.65,1.55) circle (0.07cm);
			%\draw[clip] (2.6645,4.1145) circle (0.07cm);
		\end{tikzpicture}		
	\item % S. 2.1 - Ex. 4 - o
		\begin{equation*}
		    	h(x) = \dfrac{x^2-1}{x-1}
		\end{equation*}	
		$D_h = \{ x \in \mathbb{R} | x \neq 1 \}$\\
		$h(x) = \dfrac{x^2-1}{x-1} = x + 1, x \neq 1$\\
		\begin{tikzpicture}[>=latex]
			\begin{axis}[
		  		axis x line=center,
		  		axis y line=center,
		  		%xtick={-4,...,4},
		  		xtick={-1,1},
		  		%ytick={-4,...,4},
		  		ytick={1,2},
		  		%yticklabels={-A$\flat$, -A, -B$\flat$, -B, -C, C, C$\sharp$, D, D$\sharp$, E},
			  	xlabel={$x$},
			  	ylabel={$y$},
			  	xlabel style={below right},
			  	ylabel style={above left},
			  	xmin=-4.5,
			  	xmax=4.5,
				ymin=-4.5,
		  		ymax=4.5]
			\addplot [mark=none,domain=-4:0.92] {x+1};
			\addplot [mark=none,domain=1.065:2] {x+1};			
			%\addplot [mark=none,domain=-0.94:3] {-1*x+1};
			\end{axis}
			%\draw[clip] (650,750) circle (0.07cm);
			\draw[dashed] (3.44,4.113) -- (4.13,4.113) ;%-- (4.185,2.8) ;%-- (4.2,2.8) ;
			\draw[dashed] (4.186,4.035) -- (4.186,2.8) ;%-- (4.2,2.8) ;
			%\fill (2.65,1.55) circle (0.07cm);
			\draw[clip] (4.186,4.113) circle (0.07cm);
		\end{tikzpicture}		
	\item % S. 2.1 - Ex. 4 - p
		\begin{equation*}
		    	g(x) = \dfrac{x^2-2x+1}{x-1}
		\end{equation*}	
		$D_g = \{ x \in \mathbb{R} | x \neq 1 \}$\\
		$h(x) = \dfrac{x^2-2x+1}{x-1} = \dfrac{(x-1)(x-1)}{(x-1)} = x - 1, x \neq 1$\\
		\begin{tikzpicture}[>=latex]
			\begin{axis}[
		  		axis x line=center,
		  		axis y line=center,
		  		%xtick={-4,...,4},
		  		xtick={1},
		  		xtick style={draw=none},
		  		%ytick={-4,...,4},
		  		ytick={-1},
		  		%yticklabels={-A$\flat$, -A, -B$\flat$, -B, -C, C, C$\sharp$, D, D$\sharp$, E},
			  	xlabel={$x$},
			  	ylabel={$y$},
			  	xlabel style={below right},
			  	ylabel style={above left},
			  	xmin=-4.5,
			  	xmax=4.5,
				ymin=-4.5,
		  		ymax=4.5]
			\addplot [mark=none,domain=-2:0.93] {x-1};
			\addplot [mark=none,domain=1.059:4] {x-1};			
			%\addplot [mark=none,domain=-0.94:3] {-1*x+1};
			\end{axis}
			%\draw[clip] (650,750) circle (0.07cm);
			%\draw[dashed] (3.45,4.11) -- (4.185,4.11) -- (4.185,2.8) ;%-- (4.2,2.8) ;
			\fill [black] (4.185,2.85) circle (0.07cm);
			\fill [white] (4.185,2.85) circle (0.055cm);
			%\draw[clip] (4.185,2.85) circle (0.07cm);
		\end{tikzpicture}		
	\item % S. 2.1 - Ex. 4 - q
		\begin{equation*}
		    	g(x) = \dfrac{\vert x \vert}{x}
		\end{equation*}	
		$D_g = \{ x \in \mathbb{R} | x \neq 0 \}$\\
		\begin{tikzpicture}[>=latex]
			\begin{axis}[
		  		axis x line=center,
		  		axis y line=center,
		  		%xtick={-4,...,4},
		  		xtick={0},
		  		ytick={1},
		  		ytick style={draw=none},
		  		yticklabel style={left},
		  		%yticklabels={2\phantom{42}2},
		  		%ytick={-4,...,4},
		  		extra y ticks={-1},
		  		extra y tick style={
    				yticklabel style={right},
				},
		  		extra y tick labels={\phantom{2}$-1$},
		  		%extra y tick labels={\vspace{1.5cm}$1$, $1$},
		  		%yticklabels={-A$\flat$, -A, -B$\flat$, -B, -C, C, C$\sharp$, D, D$\sharp$, E},
			  	xlabel={$x$},
			  	ylabel={$y$},
			  	xlabel style={below right},
			  	ylabel style={above left},
			  	xmin=-4.5,
			  	xmax=4.5,
				ymin=-4.5,
		  		ymax=4.5]
			\addplot [mark=none,domain=-2:-0.08] {-1};
			\addplot [mark=none,domain=0.08:2] {1};			
			%\addplot [mark=none,domain=-0.94:3] {-1*x+1};
			\end{axis}
			%\draw[clip] (650,750) circle (0.07cm);
			%\draw[dashed] (3.45,4.11) -- (4.185,4.11) -- (4.185,2.8) ;%-- (4.2,2.8) ;
			%\fill (2.65,1.55) circle (0.07cm);
			\fill [black] (3.425,3.485) circle (0.07cm);
			\fill [white] (3.425,3.485) circle (0.055cm);
			\fill [black] (3.425,2.2215) circle (0.07cm);
			\fill [white] (3.425,2.2215) circle (0.055cm);
			%\draw[clip] (3.425,3.485) circle (0.07cm) (3.425,2.2215) circle (0.07cm);
		\end{tikzpicture}		
		
	\item % S. 2.1 - Ex. 4 - r
		\begin{equation*}
		    	g(x) =\dfrac{\vert x - 1 \vert}{x-1}
		\end{equation*}	
		$D_g = \{ x \in \mathbb{R} | x \neq 1 \}$\\
		\begin{tikzpicture}[>=latex]
			\begin{axis}[
		  		axis x line=center,
		  		axis y line=center,
		  		%xtick={-4,...,4},
		  		xtick={1},
		  		xticklabel style={below right},
		  		ytick={1},
		  		yticklabel style={left},
		  		%yticklabels={2\phantom{42}2},
		  		%ytick={-4,...,4},
		  		extra y ticks={-1},
		  		extra y tick style={
    				yticklabel style={below left},
				},
		  		extra y tick labels={\phantom{2}$-1$},
		  		%extra y tick labels={\vspace{1.5cm}$1$, $1$},
		  		%yticklabels={-A$\flat$, -A, -B$\flat$, -B, -C, C, C$\sharp$, D, D$\sharp$, E},
			  	xlabel={$x$},
			  	ylabel={$y$},
			  	xlabel style={below right},
			  	ylabel style={above left},
			  	xmin=-4.5,
			  	xmax=4.5,
				ymin=-4.5,
		  		ymax=4.5]
			\addplot [mark=none,domain=-2:0.9] {-1};
			\addplot [mark=none,domain=1.1:3] {1};			
			%\addplot [mark=none,domain=-0.94:3] {-1*x+1};
			\end{axis}
			%\draw[clip] (650,750) circle (0.07cm);
			\draw[dashed] (4.185,2.35) -- (4.185,3.35);%-- (4.2,2.8) ;
			%\fill (2.65,1.55) circle (0.07cm);
			\draw[clip] (4.185,3.475) circle (0.07cm) (4.185,2.2215) circle (0.07cm);
		\end{tikzpicture}		
	\item % S. 2.1 - Ex. 4 - s
		\begin{equation*}
		    	f(x) =\dfrac{\vert 2x + 1 \vert}{2x + 1}
		\end{equation*}		
		$D_f = \{ x \in \mathbb{R} | x \neq -\dfrac{1}{2} \}$\\
		\begin{tikzpicture}[>=latex]
			\begin{axis}[
		  		axis x line=center,
		  		axis y line=center,
		  		%xtick={-4,...,4},
		  		xtick={-0.5},
		  		xticklabels={$-1/2$},
		  		xticklabel style={below left},
		  		ytick={1},
		  		yticklabel style={above right},
		  		%yticklabels={2\phantom{42}2},
		  		%ytick={-4,...,4},
		  		extra y ticks={-1},
		  		extra y tick style={
    				yticklabel style={right},
				},
		  		extra y tick labels={\phantom{2}$-1$},
		  		%extra y tick labels={\vspace{1.5cm}$1$, $1$},
		  		%yticklabels={-A$\flat$, -A, -B$\flat$, -B, -C, C, C$\sharp$, D, D$\sharp$, E},
			  	xlabel={$x$},
			  	ylabel={$y$},
			  	xlabel style={below right},
			  	ylabel style={above left},
			  	xmin=-4.5,
			  	xmax=4.5,
				ymin=-4.5,
		  		ymax=4.5]
			\addplot [mark=none,domain=-3:-0.6] {-1};
			\addplot [mark=none,domain=-0.4:3] {1};			
			%\addplot [mark=none,domain=-0.94:3] {-1*x+1};
			\end{axis}
			%\draw[clip] (650,750) circle (0.07cm);
			\draw[dashed] (3.048,2.35) -- (3.048,3.35);%-- (4.2,2.8) ;
			%\fill (2.65,1.55) circle (0.07cm);
			\draw[clip] (3.048,3.475) circle (0.07cm) (3.048,2.2215) circle (0.07cm);
		\end{tikzpicture}															    	
	\end{enumerate}
\item % S. 2.1 - Ex. 5
		$f(x) = \vert x - 1 \vert + \vert x - 2\vert$
	\begin{enumerate}

		\item % S. 2.1 - Ex. 5 - a
		Quando $x \leq 1$ temos:\\
		$x-1 \leq 0$ e $x - 2 < 0$, logo $\vert x - 1 \vert + \vert x - 2\vert = -(x-1)-(x-2)= -x+1-x+2=-2x+3$\\
				Quando $1 < x \leq 2$ temos:\\
		$x-1 > 0$ e $x - 2 \leq 0$, logo $\vert x - 1 \vert + \vert x - 2\vert = (x-1)-(x-2)= x-1-x+2=1$\\
				Quando $x > 2$ temos:\\
		$x-1 > 0$ e $x - 2 > 0$, logo $\vert x - 1 \vert + \vert x - 2\vert = (x-1)+(x-2)= x-1+x-2=2x-3$\\
		\begin{equation*}
		    	f(x) =
			    \begin{cases}
					\hfill -2x+3, & \text{se}\ x \leq 1 \\
					\hfill 1, & \text{se}\ 1 < x \leq 2 \\
					\hfill 2x-3, & \text{se}\ x > 2 \\
		    	\end{cases}
		\end{equation*}
		\item % S. 2.1 - Ex. 5 - b
		~\\		
		\begin{tikzpicture}[>=latex]
			\begin{axis}[
		  		axis x line=center,
		  		axis y line=center,
		  		%xtick={-4,...,4},
		  		xtick={1,2,3},
		  		xticklabels={$1$,$2$,$3$},
		  		xticklabel style={below},
		  		ytick={1,3},
		  		yticklabel style={left},
		  		%yticklabels={2\phantom{42}2},
		  		%ytick={-4,...,4},
		  		%extra y tick labels={\vspace{1.5cm}$1$, $1$},
		  		%yticklabels={-A$\flat$, -A, -B$\flat$, -B, -C, C, C$\sharp$, D, D$\sharp$, E},
			  	xlabel={$x$},
			  	ylabel={$y$},
			  	xlabel style={below right},
			  	ylabel style={above left},
			  	xmin=-4.5,
			  	xmax=4.5,
				ymin=-4.5,
		  		ymax=4.5]
			\addplot [mark=none,domain=-0.5:1] {-2*x+3};
			\addplot [mark=none,domain=1:2] {1};
			\addplot [mark=none,domain=2:3.5] {2*x-3};			
			%\addplot [mark=none,domain=-0.94:3] {-1*x+1};
			\end{axis}
			%\draw[clip] (650,750) circle (0.07cm);
			\draw[dashed] (4.185,2.8) -- (4.185,3.48) -- (3.4,3.48);
			\draw[dashed] (4.9525,2.8) -- (4.9525,3.48);
			\draw[dashed] (5.713,2.8) -- (5.713,4.745) -- (3.4,4.745);
			%\fill (2.65,1.55) circle (0.07cm);
			%\draw[clip] (3.048,3.475) circle (0.07cm) (3.048,2.2215) circle (0.07cm);
		\end{tikzpicture}															    	
	\end{enumerate}
\item % S. 2.1 - Ex. 6
	\begin{enumerate}
		\addtocounter{enumii}{2}
		\item % S. 2.1 - Ex. 6 - c
		$y = \vert	\vert x \vert  -1 \vert$\\
		\begin{tikzpicture}[>=latex]
			\begin{axis}[
		  		axis x line=center,
		  		axis y line=center,
		  		%xtick={-4,...,4},
		  		xtick={-1,1},
		  		xticklabels={$-1$,$1$},
		  		xticklabel style={below},
		  		ytick={1},
		  		yticklabel style={left},
		  		%yticklabels={2\phantom{42}2},
		  		%ytick={-4,...,4},
		  		%extra y tick labels={\vspace{1.5cm}$1$, $1$},
		  		%yticklabels={-A$\flat$, -A, -B$\flat$, -B, -C, C, C$\sharp$, D, D$\sharp$, E},
			  	xlabel={$x$},
			  	ylabel={$y$},
			  	xlabel style={below right},
			  	ylabel style={above left},
			  	xmin=-4.5,
			  	xmax=4.5,
				ymin=-4.5,
		  		ymax=4.5]
			\addplot [mark=none,domain=-3:3] {abs(abs(x)-1)};
			%\addplot [mark=none,domain=1:2] {1};
			%\addplot [mark=none,domain=2:3.5] {2*x-3};			
			%\addplot [mark=none,domain=-0.94:3] {-1*x+1};
			\end{axis}
			%\draw[clip] (650,750) circle (0.07cm);
			%\draw[dashed] (4.185,2.8) -- (4.185,3.48) -- (3.4,3.48);
			%\draw[dashed] (4.9525,2.8) -- (4.9525,3.48);
			%\draw[dashed] (5.713,2.8) -- (5.713,4.745) -- (3.4,4.745);
			%\fill (2.65,1.55) circle (0.07cm);
			%\draw[clip] (3.048,3.475) circle (0.07cm) (3.048,2.2215) circle (0.07cm);
		\end{tikzpicture}	
		\item % S. 2.1 - Ex. 6 - d
		$f(x)=\vert x + 1 \vert - \vert x \vert$\\
		\begin{tikzpicture}[>=latex]
			\begin{axis}[
		  		axis x line=center,
		  		axis y line=center,
		  		%xtick={-4,...,4},
		  		xtick={-1},
		  		xticklabels={$-1$},
		  		xticklabel style={above left},
		  		ytick={-1},
		  		yticklabel style={right},
		  		yticklabels={\phantom{2}-1},
		  		%ytick={-4,...,4},
		  		extra y ticks = {1},
		  		extra y tick labels={\phantom{2}1},
		  		extra y tick style={
    				yticklabel style={above right},
				},
		  		%yticklabels={-A$\flat$, -A, -B$\flat$, -B, -C, C, C$\sharp$, D, D$\sharp$, E},
			  	xlabel={$x$},
			  	ylabel={$y$},
			  	xlabel style={below right},
			  	ylabel style={above left},
			  	xmin=-4.5,
			  	xmax=4.5,
				ymin=-4.5,
		  		ymax=4.5]
			\addplot [mark=none,domain=-3:3] {abs(x+1)-abs(x)};
			%\addplot [mark=none,domain=1:2] {1};
			%\addplot [mark=none,domain=2:3.5] {2*x-3};			
			%\addplot [mark=none,domain=-0.94:3] {-1*x+1};
			\end{axis}
			%\draw[clip] (650,750) circle (0.07cm);
			%\draw[dashed] (4.185,2.8) -- (4.185,3.48) -- (3.4,3.48);
			%\draw[dashed] (4.9525,2.8) -- (4.9525,3.48);
			%\draw[dashed] (5.713,2.8) -- (5.713,4.745) -- (3.4,4.745);
			%\fill (2.65,1.55) circle (0.07cm);
			%\draw[clip] (3.048,3.475) circle (0.07cm) (3.048,2.2215) circle (0.07cm);
		\end{tikzpicture}	
		\end{enumerate}
\item % S. 2.1 - Ex. 7
\item % S. 2.1 - Ex. 8
\item % S. 2.1 - Ex. 9
\item % S. 2.1 - Ex. 10
\item % S. 2.1 - Ex. 11
\item % S. 2.1 - Ex. 12
\item % S. 2.1 - Ex. 13
\item % S. 2.1 - Ex. 14
\item % S. 2.1 - Ex. 15
\item % S. 2.1 - Ex. 16
\item % S. 2.1 - Ex. 17
\item % S. 2.1 - Ex. 18
\item % S. 2.1 - Ex. 19
\item % S. 2.1 - Ex. 20
\item % S. 2.1 - Ex. 21
\item % S. 2.1 - Ex. 22
\item % S. 2.1 - Ex. 23
\item % S. 2.1 - Ex. 24
\item % S. 2.1 - Ex. 25
\item % S. 2.1 - Ex. 26
\item % S. 2.1 - Ex. 27
\item % S. 2.1 - Ex. 28
\item % S. 2.1 - Ex. 29
\item % S. 2.1 - Ex. 30
\item % S. 2.1 - Ex. 31
\item % S. 2.1 - Ex. 32
\item % S. 2.1 - Ex. 33
\item % S. 2.1 - Ex. 34
\item % S. 2.1 - Ex. 35
\item % S. 2.1 - Ex. 36
\item % S. 2.1 - Ex. 37
\item % S. 2.1 - Ex. 38
\item % S. 2.1 - Ex. 39
\item % S. 2.1 - Ex. 40
\item % S. 2.1 - Ex. 41
\item % S. 2.1 - Ex. 42
\item % S. 2.1 - Ex. 43
\item % S. 2.1 - Ex. 44
\item % S. 2.1 - Ex. 45
\end{enumerate}