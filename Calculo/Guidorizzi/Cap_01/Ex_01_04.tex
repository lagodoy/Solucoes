\section{Intervalos}
\begin{enumerate}
\item % S. 1.4 - Ex. 1
	\begin{enumerate}
		\item % S. 1.4 - Ex. 1 - a
		$4x - 3 < 6x + 2 \Leftrightarrow 4x -6x < 2+3 \Leftrightarrow -2x < 5 \Leftrightarrow x > -\dfrac{5}{2}$\\
		$\lbrace x \in \mathbb{R}\ |\ 4x-3<6x+2 \rbrace\ =\  ]-\dfrac{5}{2},+\infty [$
		\item % S. 1.4 - Ex. 1 - b
		$\vert x \vert < 1 \Leftrightarrow -1 < x < 1$\\
		$\lbrace x \in \mathbb{R}\ |\ \vert x \vert < 1 \rbrace\ =\ ] -1, 1 [$
		\item % S. 1.4 - Ex. 1 - c
		$\vert 2x - 3 \vert \leq 1 \Leftrightarrow -1 \leq 2x - 3 \leq 1 \Leftrightarrow -1+3 \leq 2x \leq 1 + 3 \Leftrightarrow$\\
		$\Leftrightarrow 2 \leq 2x \leq 4 \Leftrightarrow 1 \leq x \leq 2$\\
		$\lbrace x \in \mathbb{R}\ |\ \vert 2x -3 \vert \leq 1 \rbrace\ =\ [1,2]$
		\item % S. 1.4 - Ex. 1 - d
		$3x+1<\dfrac{x}{3} \Leftrightarrow 3(3x+1)<x \Leftrightarrow 9x+3<x \Leftrightarrow  8x<-3\Leftrightarrow x<-\dfrac{3}{8}$\\
		$\lbrace x \in \mathbb{R}\ \vert\ 3x+1 < \dfrac{x}{3} \rbrace \ =\ ]-\infty,\dfrac{3}{8}[$
	\end{enumerate}
\item % S. 1.4 - Ex. 2
	$4-r \geq 2$, $4+r \leq 5$ e $r >0$\\
	$4 - r \geq 2 \Leftrightarrow -r \geq -2 \Leftrightarrow r \leq 2$\\
	$4+r \leq 5 \Leftrightarrow r \leq 1$, logo $0 < r \leq 1$.
\item % S. 1.4 - Ex. 3
	$p-r \geq a$ e $p+r \leq b$ com $a < b$\\
	$p-r \geq a \Leftrightarrow -r \geq a - p \Leftrightarrow r \leq p - a$\\
	$p+r \leq b \Leftrightarrow r \leq b - p$\\
	Como $r > 0$, tem-se $0 < r \leq p -a$ ou $0 < r \leq b-p$.\\
	r deve ser no máximo o menor valor dentre $p-a$ e $b-p$ para que não se ultrapasse o intervalo $]a,b[$	.
\item % S. 1.4 - Ex. 4
	\begin{enumerate}
		\item % S. 1.4 - Ex. 4 - a
		$x^2-3x+2<0 \Leftrightarrow (x-2)(x-1)<0 \Leftrightarrow 1 < x < 0$.\\
		O conjunto solução da inequação é $]1,2[$
		\item % S. 1.4 - Ex. 4 - b
		$\dfrac{2x-1}{x+3} >0 \Leftrightarrow 2x-1 > 0$ e $x +3 > 0$ ou $2x-1<0$ e $x+3<0$, logo:\\
		$x>\dfrac{1}{2}$ e $x > -3$ ou $x<\dfrac{1}{2}$ e $x < -3$.\\
		O conjunto solução da inequação é representado por:\\
		$]\dfrac{1}{2},\ +\infty [$ e $]\ -\infty, -3[$.\\
		Obs.: Há algum erro no enunciado ou na solução apresentada no livro.
		\item % S. 1.4 - Ex. 4 - c
		$x^2+x+1>0$ para todos os números reais, logo a solução é:\\ $]\ -\infty,\ +\infty\ [$.
		\item % S. 1.4 - Ex. 4 - d
		$x^2-9\leq 0 \Leftrightarrow x^2-3^2 \leq 0 \Leftrightarrow (x-3)(x+3) \leq 0 \Leftrightarrow -3 \leq x \leq 3$.\\
		O conjunto solução da inequação é $]-3,3[$		
	\end{enumerate}
\end{enumerate}

\section{Propriedades dos Intervalos Encaixantes e Propriedade de Arquimedes}
\section{Existência de raízes}