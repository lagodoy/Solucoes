\documentclass[10pt]{book}
\newcommand{\dropsign}[1]{\smash{\llap{\raisebox{-.5\normalbaselineskip}{$#1$\hspace{2\arraycolsep}}}}}%
\usepackage[utf8]{inputenc}
\usepackage{amsmath}
\usepackage{amssymb}
\usepackage{empheq}
\usepackage{enumerate}
\usepackage{setspace} % espacamento entre linhas
% padrao 1.5 de espacamento entre linhas
\setstretch{1.5}
\begin{document}
\author{Leonardo}
\title{Soluções - Guidorizzi - Volume 1}
\maketitle 
\chapter{Números Reais}
\section{Os Números Racionais}
\section{Os Números Reais}
Utilizamos nos exercícios a seguir o algoritmo de Briot-Ruffini.

\begin{enumerate}
	\setcounter{enumi}{0}
	\item
		%\begin{enumerate}[(a)] % (a), (b), (c), ...
		\begin{enumerate}\addtocounter{enumii}{3}
		\item
			$x + 3 \leq 6x - 2 \Rightarrow -5x \leq -5 \Rightarrow x \geq 1$.
		\item
			$1 - 3x > 0 \Rightarrow -3x > -1 \Rightarrow 3x < 1 \Rightarrow x < \dfrac{1}{3}$.
		\item
		$2x + 1 \geq 3x \Rightarrow -x \geq -1 \Rightarrow x \leq 1$.
		\end{enumerate}
	\item
		\begin{enumerate}\addtocounter{enumii}{8}
		\item
		$(2x - 1)(3 - 2x)$\\
		$2x - 1 < 0 \Rightarrow x < \dfrac{1}{2}$\\\\
		$2x - 1 > 0 \Rightarrow x > \dfrac{1}{2}$\\\\
		$2x - 1 = 0 \Rightarrow x = \dfrac{1}{2}$\\\\
		$3 - 2x < 0 \Rightarrow x > \dfrac{3}{2}$\\\\
		$3 - 2x > 0 \Rightarrow x < \dfrac{3}{2}$\\\\
		$3 - 2x = 0 \Rightarrow x = \dfrac{3}{2}$\\\\
		$(2x - 1)(3 - 2x) > 0$ para $\dfrac{1}{2} < x < \dfrac{3}{2}$\\
		$(2x - 1)(3 - 2x) < 0$ para $x < \dfrac{1}{2} $ ou $x > \dfrac{3}{2}$\\
		$(2x - 1)(3 - 2x) > 0$ para $\dfrac{1}{2} < x < \dfrac{3}{2}$\\
		$(2x - 1)(3 - 2x) < 0$ para $x = \dfrac{1}{2}$ ou $x = \dfrac{3}{2}$
		\item
		$x(x - 3)$\\
		$x - 3 < 0 \Rightarrow x < 3$ \\
		$x - 3 > 0 \Rightarrow x > 3$ \\
		$x - 3 = 0 \Rightarrow x = 3$ \\
		\\
		$x(x - 3) < 0$ para $0 < x < 3$\\
		$x(x - 3) > 0$ para $x < 0$ ou $x > 3$\\
		$x(x - 3)$ para $x = 0$ ou $x = 3$.
		\addtocounter{enumii}{1}
		\item
		$x(x - 1)(2x + 3)$\\
		$x - 1 < 0 \Rightarrow x < 1$\\
		$x - 1 > 0 \Rightarrow x > 1$\\
		$x - 1 = 0 \Rightarrow x = 1$\\		
		$2x + 3 < 0 \Rightarrow x < -\dfrac{3}{2}$\\\\
		$2x + 3 > 0 \Rightarrow x > -\dfrac{3}{2}$\\\\
		$2x + 3 = 0 \Rightarrow x = -\dfrac{3}{2}$\\\\
		\\
		$x(x - 1)(2x + 3) > 0$ para $-\dfrac{3}{2} < x < 0$ ou para $x > 1$\\
		$x(x - 1)(2x + 3) < 0$ para $x < -\dfrac{3}{2}$ ou para $0 < x < 1$\\
		$x(x - 1)(2x + 3) = 0$ para $x = 0$ ou $x = -\dfrac{3}{2}$ ou $x = 1$.
		\item
		$(x - 1)(1 + x)(2 - 3x)$\\
		$x - 1 < 0 \Rightarrow x < 1$\\
		$x - 1 > 0 \Rightarrow x > 1$\\
		$x - 1 = 0 \Rightarrow x = 1$\\
		$1 + x < 0 \Rightarrow x < -1$\\
		$1 + x > 0 \Rightarrow x > -1$\\
		$1 + x = 0 \Rightarrow x = -1$\\[6pt]
		$2 - 3x < 0 \Rightarrow x > \dfrac{2}{3}$\\[6pt]
		$2 - 3x > 0 \Rightarrow x < \dfrac{2}{3}$\\[6pt]
		$2 - 3x = 0 \Rightarrow x = \dfrac{2}{3}$\\[6pt]
		$(x - 1)(1 + x)(2 - 3x) > 0$ para $x < -1$ ou $\dfrac{2}{3} < x < 1$\\[6pt]
		$(x - 1)(1 + x)(2 - 3x) < 0$ para $-1 < x < \dfrac{2}{3}$ ou $x > 1$\\[6pt]
		$(x - 1)(1 + x)(2 - 3x) = 0$ para $x = 1$ ou $x = -1$ ou $x = \dfrac{2}{3}$.
		\item
		$x(x^2 + 3)$\\
		$x^2 + 3 < 0 \Rightarrow x^2 < -3$, não é possível.\\
		$x^2 + 3 > 0$ para qualquer $x$\\
		Logo temos: \\
		$x(x^2 + 3) > 0$ para $x > 0$\\
		$x(x^2 + 3) < 0$ para $x < 0$\\
		$x(x^2 + 3) = 0$ para $x = 0$\\		
		\end{enumerate}
	\item
		\begin{enumerate}\addtocounter{enumii}{7}
			\item
			$\dfrac{2x - 1}{x - 3} > 5 \Rightarrow \dfrac{2x - 1}{x - 3} > \dfrac{5(x - 3}{x - 3}\Rightarrow $\\[6pt]
			$\dfrac{2x - 1 - 5(x - 3)}{x - 3} > 0 \Rightarrow \dfrac{2x - 1 - 5x + 15}{x - 3} > 0 \Rightarrow $\\[6pt]
			$\dfrac{-3x + 14}{x - 3} > 0 \Rightarrow \dfrac{3x + 14}{x - 3} < 0$\\[6pt]
			$3x + 14 > 0 \Rightarrow x > \dfrac{14}{3}$\\[6pt]
			$3x + 14 < 0 \Rightarrow x < \dfrac{14}{3}$\\[6pt]
			$x - 3 > 0 \Rightarrow x > 3$\\
			$x - 3 < 0 \Rightarrow x < 3$\\
			Devemos ter o denominador e numerador com sinais opostos, assim temos a solução:\\
			$S =  \lbrace x \in \mathbb{R} \ |\ 3 < x < \dfrac{14}{3} \rbrace$.
			\item
			$\dfrac{x}{2x - 3}\leq 3 \Rightarrow \dfrac{x}{2x - 3} \leq \dfrac{3(2x - 3)}{2x - 3} \Rightarrow$\\[6pt]
			$\dfrac{x - 3(2x - 3)}{2x - 3} \leq 0 \Rightarrow \dfrac{x - 6x + 9}{2x - 3} \leq 0 \Rightarrow $\\[6pt]
			$\dfrac{-5x + 4}{2x - 3} \leq 0 \Rightarrow \dfrac{5x - 9}{2x - 3} \geq 0 $\\[6pt]
			$5x - 9 \geq 0 \Rightarrow x \geq \dfrac{9}{5}$\\[6pt]
			$5x - 9 < 0 \Rightarrow x < \dfrac{9}{5}$\\[6pt]
			$2x - 3 \geq 0 \Rightarrow x \geq \dfrac{3}{2}$\\[6pt]
			$2x - 3 < 0 \Rightarrow x < \dfrac{3}{2}$\\[6pt]
			$\dfrac{5x - 9}{2x - 3} \geq 0$  quando os sinais do denominador e numerador são iguais:\\[6pt]
			$S =  \lbrace x \in \mathbb{R} \ |\ x < \dfrac{3}{2}$ ou $x \geq \dfrac{9}{5}\rbrace$.
			\item
			$\dfrac{x - 1}{2 - x} < 1 \Rightarrow \dfrac{x - 1}{2 - x} < \dfrac{2 - x}{2 - x}\Rightarrow$\\[6pt]
			$\dfrac{x - 1 - 2 + x}{2 - x} < 0 \Rightarrow \dfrac{2x - 3}{2 - x} < 0$\\[6pt]
			$2x - 3 > 0 \Rightarrow x > \dfrac{3}{2}$\\[6pt]
			$2x - 3 < 0 \Rightarrow x < \dfrac{3}{2}$\\[6pt]
			$2 - x > 0 \Rightarrow x < 2$\\
			$2 - x < 0 \Rightarrow x > 2$\\	
			$\dfrac{2x - 3}{2 - x} < 0$ quando os sinais do denominador e numerador são diferentes:\\[6pt]
			$S =  \lbrace x \in \mathbb{R} \ |\ x < \dfrac{3}{2}$ ou $x > 2\rbrace$.
			\addtocounter{enumii}{1}
			\item
			$x(2x - 1)(x + 1) > 0$\\
			$2x - 1 > 0 \Rightarrow x > \dfrac{1}{2}$\\[6pt]
			$2x - 1 < 0 \Rightarrow x < \dfrac{1}{2}$\\[6pt]
			$x + 1 > 0 \Rightarrow x > -1$\\
			$x + 1 < 0 \Rightarrow x < -1$\\
			Os valores para que tenhamos $x(2x - 1)(x + 1) > 0$:\\
			$S =  \lbrace x \in \mathbb{R} \ |\ -1 < x < 0$ ou $x > \dfrac{1}{2} \rbrace$.
			\item
			$(2x - 1)(x - 3) > 0$\\
			$2x - 1 > 0 \Rightarrow x > \dfrac{1}{2}$\\[6pt]
			$2x - 1 < 0 \Rightarrow x < \dfrac{1}{2}$\\[6pt]
			$x - 3 > 0 \Rightarrow x > 3$\\
			$x - 3 < 0 \Rightarrow x < 3$\\
			Devemos ter os sinais dos fatores iguais em 	$(2x - 1)(x - 3) > 0$: 
			$S =  \lbrace x \in \mathbb{R} \ |\ x < \dfrac{1}{2}$ ou $x > 3 \rbrace$.
			\item			
			$(2x - 3)(x^2 + 1) < 0$\\
			$x^2 + 1 > 0$ para qualquer $x$.\\
			Devemos ter $2x - 3 < 0 \Rightarrow x < \dfrac{3}{2}$:\\
			$S =  \lbrace x \in \mathbb{R} \ |\ x < \dfrac{3}{2} \rbrace$.
			\item
			$\dfrac{x - 3}{x^2 + 1} < 0$\\
			$x^2 + 1 > 0$ para qualquer $x$.\\
			Devemos ter $x - 3 < 0 \Rightarrow x < 3$:\\
			$S =  \lbrace x \in \mathbb{R} \ |\ x < 3 \rbrace$.
		\end{enumerate}
		\item % 4
			$x^3 + 0x^2 + 0x - a^3 \div \llcorner x - a$:
			\\
			\[
  \begin{array}{r|r}
    \dropsign{-} x^3 + 0x^2 + 0x - a^3 & x - a \\ \cline{2-2}
    x^3 - ax^2 \phantom{ + 0x - a^3}& x^2 + ax + a^2 \\ \cline{1-1} \\[\dimexpr-\normalbaselineskip+\jot]
    \dropsign{-}ax^2 + 0x - a^3 \\
           \phantom{0x^3 + } ax^2 - a^2x \phantom{+ a^3} \\ \cline{1-1} \\[\dimexpr-\normalbaselineskip+\jot]
    \dropsign{-}\phantom{ax^2 + }a^2x - a^3 \\
           \phantom{0x^3 + ax^2 + } a^2x - a^3 \\ \cline{1-1} \\[\dimexpr-\normalbaselineskip+\jot]
                      0
  \end{array}
\]
		\item % S. 1.2 - Ex. 5
		\begin{enumerate}
			\addtocounter{enumii}{2}
			\item % S. 1.2 - Ex. 5 - c
				$(x - a)(x^3 + ax^2 +a^2x + a^3) = $\\
				$x^4 + ax^3 + a^2x^2 + a^2x^3 - ax^3 - a^2x^2 - a^3x - a^4 = $\\
				$x^4 - a^4$
				
			\item % S. 1.2 - Ex. 5 - d
				$(x - a)(x^4 + ax^3 +a^2x^2 + a^3x + a^4) = $\\
				$x^5 + ax^4 + a^2x^3 + a^3x^2 + a^4x - ax^4 - a^2x^3 - a^3x^2 - a^4x - a^5 = $\\
				$x^5 - a^5$
				
			\item % S. 1.2 - Ex. 5 - e
			$(x - a)(x^{n-1} + ax^{n-2} +a^2x^{n-3} + ... + a^{n-2}x + a^{n-1}) = $\\
			$x^{n} + ax^{n-1} +a^2x^{n-2} + ... + a^{n-2}x^2 + a^{n-1}x$ \\ 
			$- ax^{n-1} - a^2x^{n-2} -a^3x^{n-3} - ... - a^{n-1}x - a^{n} =$\\
			$x^{n} - a^{n}$\\
			
		\end{enumerate}
		\item % S. 1.2 - Ex. 6
		\begin{enumerate}
			\addtocounter{enumii}{7}
			\item % S. 1.2 - Ex. 6 - h
			$\dfrac{\dfrac{1}{x}-\dfrac{1}{p}}{x - p} = \dfrac{\dfrac{p - x}{xp}}{x - p} = \dfrac{p - x}{(x - p)xp} = -\dfrac{1}{xp}$.\\
			\item % S. 1.2 - Ex. 6 - i
			$\dfrac{\dfrac{1}{x^2}-\dfrac{1}{p^2}}{x-p} = \dfrac{\dfrac{p^2-x^2}{x^2p^2}}{x-p} = \dfrac{p^2-x^2}{x^2p^2(x-p)} = \dfrac{(p-x)(p+x)}{-x^2p^2(p-x)} = -\dfrac{x+p}{x^2p^2}$.\\
			\addtocounter{enumii}{1}
			\item % S. 1.2 - Ex. 6 - j
			$\dfrac{x^4-p^4}{x-p} = \dfrac{(x-p)(x^3+px^2+p^2x+p^3}{(x-p)} = x^3+px^2+p^2x+p^3$\\
			\item % S. 1.2 - Ex. 6 - l
			$\dfrac{(x+h)^2-x^2}{h} = \dfrac{x^2+2hx+h^2-x^2}{h} =$\\
			$ \dfrac{2hx + h^2}{h} = \dfrac{h(2x + h)}{h} = 2x + h$\\
			\item % S. 1.2 - Ex. 6 - m
			$\dfrac{\dfrac{1}{x+h}-\dfrac{1}{x}}{h} = \dfrac{\dfrac{x-x-h}{x(x+h)}}{h} = \dfrac{\dfrac{-h}{x(x+h)}}{h}=\dfrac{-h}{x(x+h)h} = -\dfrac{1}{x(x+h)}$\\
			\item % S. 1.2 - Ex. 6 - n
			$\dfrac{(x+h)^3-x^3}{h} = \dfrac{x^3+3hx^2+3h^2x+h^3-x^3}{h} =$\\ 						$\dfrac{3hx^2+3h^2x+h^3}{h} = 3x^2+3xh+h^2$\\
			\item % S. 1.2 - Ex. 6 - o
			$\dfrac{(x+h)^2-(x-h)^2}{h} = \dfrac{(x^2+2hx+h^2)-(x^2-2hx+h^2)}{h} =$\\ 
			$\dfrac{4hx}{h} = 4x$\\
		\end{enumerate}
		\item % S. 1.2 - Ex. 7
		\begin{enumerate}
			\addtocounter{enumii}{5}
			\item % S. 1.2 - Ex. 7 - f
			$\dfrac{x^2 - 4}{x^2 + 4} > 0 \Rightarrow \dfrac{(x + 2)(x - 2)}{x^2 + 4} > 0$\\
			O denominador será sempre positivo, devemos analisar o sinal do numerador $(x + 2)(x - 2)$:\\
			$x + 2 > 0 \Rightarrow x > -2$\\
			$x + 2 < 0 \Rightarrow x < -2$\\
			$x - 2 > 0 \Rightarrow x > 2$\\
			$x - 2 < 0 \Rightarrow x < 2$\\
			O numerador deve ser positivo, portanto os fatores do produto devem ter o mesmo sinal, temos isso quando $x < -2$ ou $x > 2$.
			\item % S. 1.2 - Ex. 7 - g
			$(2x - 1)(x^2 - 4) \leq 0 \Rightarrow (2x - 1)(x + 2)(x - 2) \leq 0$\\
			O produto deve ter seus fatores todos negativos ou ao menos um deles com valor 0 ou, finalmente, dois positivos e um negativo. Temos então os sinais de cada expressão:\\
			$2x - 1 \leq 0 \Rightarrow x \leq \dfrac{1}{2}$\\
			$2x - 1 > 0 \Rightarrow x > \dfrac{1}{2}$\\
			$x + 2 \leq 0 \Rightarrow x \leq -2$\\
			$x + 2 > 0 \Rightarrow x > -2$\\
			$x - 2 \leq 0 \Rightarrow x \leq 2$\\
			$x - 2 > 0 \Rightarrow x > 2$\\
			Para termos $(2x - 1)(x^2 - 4) = 0$, basta termos $x = \dfrac{1}{2}$ ou $x = -2$ ou $x = 2$.\\
			Para termos $(2x - 1)(x^2 - 4) < 0$, com os três fatores acima mencionados negativos, devemos ter $x < -2$.\\
			Finalmente para termos $(2x - 1)(x^2 - 4) < 0$, com dois fatores positivos e um negativo. Analisando a seguir todas as possibilidades:\\
			$x < \dfrac{1}{2}$ e $x > -2$ e $x > 2$ não é possível.\\\\
			$x > \dfrac{1}{2}$ e $x < -2$ e $x > 2$ não é possível.\\\\
			$x > \dfrac{1}{2}$ e $x > -2$ e $x < 2$ nos dá  $\dfrac{1}{2} < x < 2$.\\\\
			Logo as soluções são  $x \leq -2$ ou $\dfrac{1}{2} \leq x \leq 2$.
			
			\item % S. 1.2 - Ex. 7 - h
			$3x^2 \geq 48 \Rightarrow x^2 \geq 16 \Rightarrow (x - 4)(x + 4) \geq 0$\\
			Estudando o sinal do produto, devemos ter as duas expressões com mesmo sinal ou alguma delas com valor zero:\\
			$x-4 > 0 \Rightarrow x > 4$\\
			$x-5 < 0 \Rightarrow x < 4$\\
			$x+5 > 0 \Rightarrow x > -4$\\
			$x+5 < 0 \Rightarrow x < -4$\\
			Devemos ter:\\
			$x > 4$ e $x > -4$, que nos dá $x > 4$.\\
			ou\\
			$x < 4$ e $x < -4$ que nos dá $x < -4$.\\
			A solução portanto é ou $x \leq -4$ ou $x \geq 4$
			\item % S. 1.2 - Ex. 7 - i
			$x^2 < r^2 \Rightarrow (x-r)(x+r) < 0$
			Estudando o sinal do produto, devemos ter as duas expressões com sinais opostos:\\
			$x-r > 0 \Rightarrow x > r$\\
			$x-r < 0 \Rightarrow x < r$\\
			$x+r > 0 \Rightarrow x > -r$\\
			$x+r < 0 \Rightarrow x < -r$\\
			Devemos ter:\\
			$x > r$ e $x < -r$, que é impossível.\\
			ou\\
			$x < r$ e $x > -r$ que nos dá $-r < x < r$.
			\item % S. 1.2 - Ex. 7 - j
			$x^2 \geq r^2 \Rightarrow (x-r)(x+r) \geq 0$
			Estudando o sinal do produto, devemos ter as duas expressões com mesmo sinal ou alguma delas com valor zero:\\
			$x-r > 0 \Rightarrow x > r$\\
			$x-r < 0 \Rightarrow x < r$\\
			$x+r > 0 \Rightarrow x > -r$\\
			$x+r < 0 \Rightarrow x < -r$\\
			Devemos ter:\\
			$x > r$ e $x > -r$, que nos dá $x > r$.\\
			ou\\
			$x < r$ e $x < -r$ que nos dá $x < -r$.\\
			A solução portanto é ou $x \leq -r$ ou $x \geq r$
		\end{enumerate}
		\item % S. 1.2 - Ex. 8
		\begin{enumerate}
			\item % S. 1.2 - Ex. 8 - a
				$a\left[ \left( x + \dfrac{b}{2a}\right)^2 - \dfrac{\left( b^2 -4ac\right)}{4a^2} \right] = a\left[ x^2 + \dfrac{2xb}{2a} + \dfrac{b^2}{4a^2}- \dfrac{b^2 + 4ac}{4a^2}\right] =$\\\\
				= $\boxed { ax^2 + bx + c}$
			\item % S. 1.2 - Ex. 8 - b
				$a\left[ \left(\dfrac{-b\pm\sqrt{\Delta}}{2a} + \dfrac{b}{2a}\right)^2 - \dfrac{\Delta}{4a^2}\right] = a\left[\left(\pm\dfrac{\sqrt{\Delta}}{2a}\right)^2 -\dfrac{\Delta}{4a^2}\right]=$\\
				$=a\left[\dfrac{\Delta}{4a^2}- \dfrac{\Delta}{4a^2}\right] = a.0 = \boxed{0}$
			\item % S. 1.2 - Ex. 8 - c
				$x_{1} + x_{2} = \dfrac{-b+\sqrt{\Delta}}{2a} + \dfrac{(-b-\sqrt{\Delta})}{2a} = \dfrac{-b+\sqrt{\Delta}-b-\sqrt{\Delta}}{2a} = $\\\\
				$=\dfrac{-2b}{2a}=\boxed{\dfrac{-b}{a}}$\\\\
				$x_{1}.x_{2} = \dfrac{-b+\sqrt{\Delta}}{2a}.\dfrac{(-b-\sqrt{\Delta})}{2a} = \dfrac{b^2-\Delta}{4a^2} = \dfrac{b^2-b^2+4ac}{4a^2} = \boxed{\dfrac{c}{a}}$
		\end{enumerate}
		\item % S. 1.2 - Ex. 9
		$a(x-x_{1})(x-x_{2}) = a(x^2 - xx_{2} - xx_{1} +x_{1}x_{2}) = $\\
		$ = a[x^2 - x(x_{2} + x_{1}) + x_{1}x_{2}] = a\left[x^2 - x\left(\dfrac{-b}{a}\right) + \dfrac{c}{a}\right] =$\\ $
		= \boxed{ax^2+bx+c}$
		\item % S. 1.2 - Ex. 10
		\begin{enumerate}
			\setcounter{enumii}{5}
				\item % S. 1.2 - Ex. 10 - f
				$2x^2-3x+1=2(x^2-\dfrac{3}{2}x+\dfrac{1}{2}) = 2(x-1)(x-\dfrac{1}{2})=\boxed{(x-1)(2x-1)}$
				\item % S. 1.2 - Ex. 10 - g
				$x^2 - 25$\\
				$x_{1} + x_{2} = 0 \Rightarrow x_{1} = -x_{2}$\\
				$x_{1}x_{2} = -25 \Rightarrow (-x_{1})x_{1} = -25 \Rightarrow x_{1}^2 = 25 \Rightarrow x_{1} = 5$ e $x_{2} = -5$\\
				$\boxed{(x - 5)(x + 5)}$
				\item % S. 1.2 - Ex. 10 - h
				$3x^2 + x - 2 = 3(x^2+\dfrac{1}{3}x-\dfrac{2}{3}) = 3(x+1)(x-\dfrac{2}{3})=\boxed{(x+1)(3x-2)}$\\
				
				
				\item % S. 1.2 - Ex. 10 - i
				$4x^2 - 9$
				$x_{1} + x_{2} = 0 \Rightarrow x_{1} = -x_{2}$\\
				$x_{1}x_{2} = -\dfrac{9}{4} \Rightarrow (-x_{1})x_{1} = -\dfrac{9}{4} \Rightarrow x_{1}^2 = \dfrac{9}{4} \Rightarrow x_{1} = \dfrac{3}{2}$ e $x_{2} = -\dfrac{3}{2}$\\
				$4(x+\dfrac{3}{2})(x-\dfrac{3}{2}) = \boxed{(2x-3)(2x+3)}$
				\item % S. 1.2 - Ex. 10 - j
				$2x^2 - 5x$\\
			 	$x_{1}x_{2} = 0 \Rightarrow x_{1} = 0$\\
			 	$x_{1} + x_{2} = \dfrac{5}{2} \Rightarrow x_{2}  = \dfrac{5}{2}$\\
			 	$2x(x-\dfrac{5}{2}) = \boxed{x(2x-5)}$
		\end{enumerate}
	\item % S. 1.2 - Ex. 11
		\begin{enumerate}
			\setcounter{enumii}{5}
			\item % S. 1.2 - Ex. 11 - f
			$x_{1}+x_{2}= -\dfrac{1}{3}$ e $x_{1}x_{2}=-\dfrac{2}{3} $\\[6pt]
			$x_{1}=\dfrac{2}{3}$ e $x_{2}=-1$\\[6pt]
			$3x^2+x-2 > 0 \Rightarrow 3(x+1)(x-\dfrac{2}{3}) > 0 \Rightarrow (x+1)(x-\dfrac{2}{3}) > 0$\\[6pt]
			Temos os seguintes sinais para cada fator do produto:\\
			$x + 1 > 0$ quando $x > -1$\\
			$x + 1 > 0$ quando $x < -1$\\
			$x-\dfrac{2}{3} > 0$ quando $x > \dfrac{2}{3}$\\[6pt]
			$x-\dfrac{2}{3} < 0$ quando $x < \dfrac{2}{3}$\\[6pt]
			Para termos $(x+1)(x-\dfrac{2}{3}) > 0$, os fatores devem possuir o mesmo sinal:\\
			$x > \dfrac{2}{3}$ ou  $x < -1$
			\item % S. 1.2 - Ex. 11 - g
			$x_{1}+x_{2}= 4$ e $x_{1}x_{2}= 4$\\
			$x_{1} = x_{2}= 2$\\
			$x^2-4x+4 > 0 \Rightarrow (x - 2)(x - 2) > 0 \Rightarrow (x-2)^2 >0$\\
			A equação é sempre positiva, exceto em $x = 2$.\\
			A solução é $x \neq 2$.
			\item % S. 1.2 - Ex. 11 - h
			$x_{1}+x_{2}=\dfrac{1}{3}$ e $x_{1}x_{2}= 0$\\[6pt]
			$x_{1}=\dfrac{1}{3}$\\[6pt]
			$3x^2-x \leq 0 \Rightarrow 3x(x-\dfrac{1}{3})\leq 0 \Rightarrow x(x-\dfrac{1}{3})\leq 0$\\[6pt]
			Temos os seguintes sinais para cada fator do produto:\\
			$x-\dfrac{1}{3} > 0 \Rightarrow x>\dfrac{1}{3}$\\[6pt]
			$x-\dfrac{1}{3} \leq 0 \Rightarrow x \leq \dfrac{1}{3}$\\[6pt]
			$x > 0$ ou $x \leq 0$\\
			Para termos $ x(x-\dfrac{1}{3})\leq 0$, os fatores devem possuir sinais opostos ou $x = 0$ ou $x = \dfrac{1}{3}$:\\[6pt]
			$0 \leq x \leq \dfrac{1}{3}$
			\item % S. 1.2 - Ex. 11 - i
			$x_{1}+x_{2}=1$ e $x_{1}x_{2}=\dfrac{1}{4}$\\[6pt]
			$x_{1} = x_{2}= \dfrac{1}{2}$\\[6pt]
			$4x^2 - 4x + 1 < 0 \Rightarrow 4(x - \dfrac{1}{2})(x - \dfrac{1}{2}) < 0 \Rightarrow (x - \dfrac{1}{2})^2 < 0$\\[6pt]
			A desigualdade não é possível para nenhum $x$.
			\item % S. 1.2 - Ex. 11 - j
			$x_{1}+x_{2}= 1$ e $x_{1}x_{2}=\dfrac{1}{4} $\\[6pt]
			$x_{1} = x_{2}= \dfrac{1}{2}$\\[6pt]
			$4x^2 - 4x + 1 \leq 0 \Rightarrow 4(x - \dfrac{1}{2})(x - \dfrac{1}{2}) \leq 0 \Rightarrow (x - \dfrac{1}{2})^2 \leq 0$\\[6pt]
			A inequação somente é possível para $x=\dfrac{1}{2}$.
		\end{enumerate}
	\item % S. 1.2 - Ex. 12
		\begin{enumerate}
			\item % S. 1.2 - Ex. 12 - a
			Por 8. a) temos:\\
			$ax^2+bx+c = a\left[\left(x+\dfrac{b}{2a}\right)^2-\dfrac{\Delta}{4a^2}\right]$\\[6pt]
			Observando o produto, do lado direito temos $a > 0$ e a expressão entre colchetes é positiva sempre, pois temos um termo elevado ao quadrado e no outro, $\dfrac{\Delta}{4a^2}$, temos $\Delta < 0$, porém precedido por um sinal negativo, e dividido por $4a^2$ que é positivo, conclui-se então que esse fator também é positivo e por consequência $ax^2+bx+c > 0$.
			\item % S. 1.2 - Ex. 12 - b
			O raciocínio é similar ao item anterior, exceto que agora temos $a < 0$ e portanto o produto $a\left[\left(x+\dfrac{b}{2a}\right)^2-\dfrac{\Delta}{4a^2}\right]$ é negativo, o que nos dá  $ax^2+bx+c < 0$.
		\end{enumerate}
	\item % S. 1.2 - Ex. 13
	\begin{enumerate}
		\setcounter{enumii}{5}
		\item % S. 1.2 - Ex. 13 -f
		$(2x + 1)(x^2+x+1) \leq 0$\\
			Estudando o sinal do produto, devemos ter as duas expressões com sinais opostos ou algum fator ser 0::\\
			$2x + 1 > 0 \Rightarrow x > -\dfrac{1}{2}$\\
			$2x + 1 < 0 \Rightarrow x \leq -\dfrac{1}{2}$\\
			De acordo com o exercício 12 acima, $x^2+x+1 >0$, pois $\Delta = 1 - 4.1.1 = -3 < 0$.\\ 
			Logo $(2x + 1)(x^2+x+1) \leq 0$ quando $x \leq -\dfrac{1}{2}$.
		\item % S. 1.2 - Ex. 13 - g
		$x(x^2+1)\geq 0$\\
		Estudando o sinal do produto, devemos ter as duas expressões com sinais iguais ou algum fator ser 0:\\
		$x^2+1$ tem $a = 1 > 0$ e $\Delta = 0 - 4.1.1=-4<0$, que pelo exercício 12 nos dá $x^2+1 > 0$.\\
		Logo devemos ter $x \geq 0$.
		\item % S. 1.2 - Ex. 13 - h
		$(1-x)(x^2+2x+2)<0$\\
		Estudando o sinal do produto, devemos ter as duas expressões com sinais opostos:\\
		$x^2+2x+2$ tem $a=1 >0$ e $\Delta = 4-4.1.2=-4 <0$, que pelo exercício 12 nos dá $x^2+2x+2 > 0$.\\
		Logo devemos ter $1-x < 0 \Rightarrow x > 1$.
		\item % S. 1.2 - Ex. 13 - i
		$\dfrac{2x-3}{x^2+1}>0$\\		
		Estudando o sinal da divisão, devemos ter as duas expressões com sinais iguais:
		$x^2+1$ tem $a = 1 > 0$ e $\Delta = 0 - 4.1.1 = -4 < 0$, que pelo exercício 12 nos dá $x^2+1 > 0$.\\
		Portanto devemos ter $2x - 3 > 0 \Rightarrow x > \dfrac{3}{2}$.
		\item % S. 1.2 - Ex. 13 - j
		$\dfrac{x}{x^2+x+1}\geq 0$\\
		Estudando o sinal da divisão, devemos ter as duas expressões com sinais iguais ou $x = 0$:\\
		$x^2+x+1$ tem $a=1 > 0$ e $\Delta = 1-4.1.1 = -3 < 0$, que pelo exercício 12 nos dá $x^2+x+1 > 0$.\\
		Portanto devemos ter $x \geq 0$
	\end{enumerate}
	\item % S. 1.2 - Ex. 14
	Primeiramente observa-se o fato de $x^2+1$ ser sempre positivo, portanto ao multiplicarmos ambos os lados da expressão $\dfrac{5x+3}{x^2+1} \geq 5$ por $x^2 + 1$ a direção da desigualdade não se altera:\\[6pt]
	$\dfrac{5x+3}{x^2+1} \geq 5 \Rightarrow \dfrac{5x+3}{x^2+1}.(x^2+1) \geq 5(x^2+1) \Rightarrow 5x+3 \geq 5(x^2+1)$.\\[6pt]
	Por outro lado dividimos $5x+3 \geq 5(x^2+1)$ por $x^2+1$ e temos:\\[6pt]
	$\dfrac{5x+3}{x^2+1} \geq 5\dfrac{(x^2+1)}{(x^2+1)} \Rightarrow \dfrac{5x+3}{x^2+1} \geq 5$.
	\setcounter{enumi}{16}
	\item % S. 1.2 - Ex. 17
		\begin{enumerate}
			\setcounter{enumii}{3}
			\item  % S. 1.2 - Ex. 17 - d
			$2x^3-x^2-1=0$\\
			$1$ e $-1$ são os divisores de $a_{3}=-1$, testando os dois valores na equação temos $1$ como raiz inteira da equação.
			\item  % S. 1.2 - Ex. 17 - e
			$x^3+x^2+x-14=0$\\
			Os divisores inteiros de $a_{4}-14$ são $\pm1$, $\pm2$, $\pm7$ e $\pm14$.\\
			Testando as 8 possibilidades temos:\\
			$1^3+1^2+1-14=-11$\\
			$-1^3+(-1)^2-1-14=-15$\\
			$2^3+2^2+2-14=0$, logo $2$ é raiz.\\
			$-2^3+(-2)^2-2-14=-12$\\
			$7^3+7^2+7-14=385$\\
			$-7^3+(-7)^2-7-14=-315$\\
			$14^3+14^2+14-14=2940$\\
			$-14^3+(-14)^2-14-14=-2576$\\
			A única raiz inteira encontrada é $2$.
			\item  % S. 1.2 - Ex. 17 - f
			$x^3+3x^2-4x-12=0$\\
			Os divisores inteiros de $a_{4}-12$ são $\pm1$, $\pm2$, $\pm3$, $\pm4$, $\pm6$ e $\pm12$.\\
			Testando as 12 possibilidades temos:\\
			$1^3+3(1)^2-4(1)-12=-12$\\
			$-1^3+3(-1)^2-4(-1)-12=-6$\\
			$2^3+3(2)^2-4(2)-12=0$, logo 2 é raiz\\
			$-2^3+3(-2)^2-4(-2)-12=0$, logo -2 é raiz.\\
			$3^3+3(3)^2-4(3)-12=30$\\
			$-3^3+3(-3)^2-4(-3)-12=0$, logo -3 é raiz.\\
			$4^3+3(4)^2-4(4)-12=84$\\
			$-4^3+3(-4)^2-4(-4)-12=-12$\\
			$6^3+3(6)^2-4(6)-12=288$\\
			$-6^3+3(-6)^2-4(-6)-12=-96$\\
			$12^3+3(12)^2-4(12)-12=2100$\\
			$-12^3+3(-12)^2-4(-12)-12=-1260$\\
			As raízes inteiras são 2, -2 e -3.
		\end{enumerate}
	\setcounter{enumi}{18}
	\item % S. 1.2 - Ex. 19
	\begin{enumerate}
		\item % S. 1.2 - Ex. 19 - a
			$x^3 + 2x^2 - x - 2$
			\\
	 		\\
			\begin{tabular}{ l | r r r r }
			  1 & 1 & 2 & -1 & -2 \\
				\hline
			    & 1 & 3 & 2 & 0 \\
			\end{tabular}
			\\
			\\
			$(x^2 + 3x + 2)(x - 1)$
			\\
			\\
			\begin{tabular}{ l | r r r }
			  -2 & 1 & 3 & 2 \\
			  \hline
			    & 1 & 1 & 0  \\
			\end{tabular}
			\\
			\\
		   $(x - 1)(x + 1)(x + 2)$.
		   \\
		\item % S. 1.2 - Ex. 19 - b
			$x^4 - 3x^3 + x^2 + 3x - 2$
			\\
			\\
			\begin{tabular}{ l | r r r r r }
			   -1 & 1 & -3 & 1 & 3 & -2  \\
				\hline
			    & 1 & -4 & 5 & -2 & 0 \\
			\end{tabular}
			\\
			\\
			$(x^3 - 4x^2 + 5x - 2)(x + 1)$
	 		\\
	 		\\
			\begin{tabular}{ l | r r r r }
			   1 & 1 & -4 & 5 & -2 \\
				\hline
			    & 1 & -3 & 2 & 0 \\
			\end{tabular}
			\\
			\\
			$(x + 1)(x - 1)(x^2 - 3x + 2)$
			\\
			\\
			\begin{tabular}{ l | r r r r }
			   1 & 1 & -3 & 2 \\
				\hline
			    & 1 & -2 & 0  \\
			\end{tabular}
			\\
			\\
			$(x + 1)(x - 1)^2(x - 2)$.
			\\
		\item % S. 1.2 - Ex. 19 - c
			$x^3 + 2x^2 - 3x$
			\\
			$(x^2 + 2x - 3)x$
			\\
			\\
			\begin{tabular}{ l | r r r r }
			  1 & 1 & 2 & -3 \\
				\hline
			    & 1 & 3 & 0 \\
			\end{tabular}
			\\
			\\
			$(x + 3)(x - 1)x$.
			\\
			\item % S. 1.2 - Ex. 19 - d
			$x^3+3x^2-4x-12$
			\\
			\\
			\begin{tabular}{ l | r r r r r }
			  -2 & 1 & 3 & -4 & -12 \\
				\hline
			    & 1 & 1 & -6 & 0 \\
			\end{tabular}
			\\
			\\
			$(x^2+x-6)(x+2)$
			\\
			\\
			\begin{tabular}{ l | r r r r }
			  2 & 1 & 1 & -6 \\
				\hline
			    & 1 & 3 & 0 \\
			\end{tabular}
			\\
			$(x+3)(x-2)(x+2)$.
			\item % S. 1.2 - Ex. 19 - e
			$x^3+6x^2+11x+6$
			\\
			\\
			\begin{tabular}{ l | r r r r r}
			  -1 & 1 & 6 & 11 & 6\\
				\hline
			    & 1 & 5 & 6 & 0 \\
			\end{tabular}
			\\
			\\
			$(x^2+5x+6)(x+1)$
			\\
			\\
			\begin{tabular}{ l | r r r r }
			  -2 & 1 & 5 & 6 \\
				\hline
			    & 1 & 3 & 0 \\
			\end{tabular}
			\\
			\\
			$(x+2)(x+1)(x+3)$.
			\item % S. 1.2 - Ex. 19 - f
			$x^3-1$
			\\
			\\
			\begin{tabular}{ l | r r r r }
			  1 & 1 & 0 & 0 & -1 \\
				\hline
			    & 1 & 1 & 1 & 0 \\
			\end{tabular}
			\\
			\\
			$(x^2+x+1)(x-1)$
	\end{enumerate}
	\item % S. 1.2 - Ex. 20
		\begin{enumerate}
		\item % S. 1.2 - Ex. 20 - a
			$x^3 - 1 > 0$\\
			$x^3 > 1 \Rightarrow x > 1$
		\item  % S. 1.2 - Ex. 20 - b
			$x^3 + 6x^2 + 11x + 6 < 0$
			\\
			\\
			\begin{tabular}{ l | r r r r r }
			  -1 & 1 & 6 & 11 & 6 \\
				\hline
			    & 1 & 5 & 6 & 0 \\
			\end{tabular}		
			\\
			\\
			$(x + 1)(x^2 + 5x + 6) < 0$
			\\
			\\
			\begin{tabular}{ l | r r r r }
			  -2 & 1 & 5 & 6 \\
				\hline
			    & 1 & 3 & 0 \\
			\end{tabular}	
			\\
			\\
			$(x + 1)(x + 2)(x + 3) < 0$\\
			Basta estudarmos o sinal da última inequação, onde deveremos ter um número ímpar de elementos do produto negativos:\\
			Com $x < -3$, $x < -2$ e $x < -1$ temos $x < -3$.\\
			
			Nos casos com apenas um elemento negativo:\\
			\begin{list}{•}
			\item
			$(x + 1) < 0$ nos dá $x < -1$ e devemos ter $x > -2, x > -3$, o que nos dá $x > -2$.
			\item
			$(x + 2) < 0$ nos dá $x < -2$ e devemos ter $x > -1, x > -3$, o que nao é possível.
			\item
			$(x + 3) < 0$ nos dá $x < -3$ e devemos ter $x > -1, x > -2$, mas não existe tal combinação.
			\end{list}
			Finalmente temos a outra solução da inequação:\\
			$-2 < x < -1$.

			É possível, e até mais prático, estudar os sinais acima graficamente.
		\item  % S. 1.2 - Ex. 20 - c
			$x^3 + 3x - 4x - 12 \geq 0 \Rightarrow x(x - 1)(x + 3) < 0$ \\
			
			Estudamos a seguir o sinal da inequação $x(x - 1)(x + 3) < 0$, devemos ter um número ímpoar de elementos do produto negativos: \\
			Com $x < 0$, $x - 1 < 0 \Rightarrow x < 1$ e $x + 3 < 0 \Rightarrow x < -3$, temos $x < -3$.\\
			Nos casos com apenas um elemento negativo:
			\begin{list}{•}
			\item
			Com $x < 0$ devemos ter $x - 1 > 0 $ e $x + 3 > 0$, portanto respectivamente $x > 1$ e $x > -3$, mas não existe tal combinação.
			\item\item
			$x - 1 < 0$ nos dá $x < 1$ e devemos ter $x > 0$  e $x > -3$, o que resulta em $0 < x < 1$
			\item
			$x + 3 < 0$ nos dá $x < -3$ e devemos ter $x > 0$ e $x > 1$, mas não é possível tal combinação.
			\end{list}
			As solução da inequação é $x < -3$ ou $0 < x < 1$.
		\item  % S. 1.2 - Ex. 20 - d
			$x^3 + 2x^2 - 3x < 0$
		\end{enumerate}
	\item
	Falsa. Para explicar basta darmos um contra-exemplo: 
	\paragraph{}Se $x = -1$ e $y = 0$, temos $x < y$, mas não $x^2 < y^2$, pois daí teríamos $1 < 0$, o que contradiz nossa proposição.
	\item 
	$x^3  - y^3 = (x - y)(x^2 + xy + y^2)$\\
	Se x e y têm o mesmo sinal, temos $(x^2 + xy + y^2) > 0$.
	
	Portanto devemos estudar o que ocorre quando $x - y < 0$, com a condição de x e y terem o mesmo sinal.\\
	Se $x > 0$ e $y > 0$ temos $x - y < 0 \Rightarrow x < y$.\\
	No caso de $x < 0$ e $y < 0$ temos, de forma similar, $x - y < 0 \Rightarrow x < y$.\\
	Já quando ocorrem sinais opostos para $x$ e $y$, temos apenas da avaliar o caso em que $x < 0$ e $y > 0$.\\
	Temos então:\\
	$x^3 < 0 \Rightarrow x^2 > 0 \Rightarrow x < 0$ (a ordem da desigualdade vai sendo trocada em cada produto pelo inverso) e\\
	$y^3 > 0 \Rightarrow y^2 > 0 \Rightarrow y > 0$ (a ordem permanece intacta em cada produto pelo inverso).
	
	Finalmente pela lei da transitividade temos $x < 0$ e $0 < y \Rightarrow x < y$.
	\\\\
	Por outro lado:\\
	$x < y \Rightarrow x - y < 0$\\
	$x > 0$ e $y > 0 \Rightarrow x^2 + xy + y^2 > 0$ e\\
    $x < 0$ e $y < 0	\Rightarrow  x^2 + xy + y^2 > 0$
	
	Multiplicando-se os dois lados da inequação $x - y < 0$ por $x^2 + xy + y^2$ conserva a ordem da desigualdade:\\
	$(x - y)(x^2 + xy + y^2) < 0 \Rightarrow x^3 - y^3 < 0 \Rightarrow x^3 < y^3$.\\
	Caso tenhamos x e y com sinais diferentes, pegamos apenas o caso em que $x < 0$ e $y > 0$, pois o contrário não existe para $x < y$.\\
	Temos então:\\
	$x < 0 \Rightarrow x^2 > 0 \Rightarrow x^3 < 0$ (a ordem da desigualdade vai sendo trocada em cada produto) e\\
	$y > 0 \Rightarrow y^2 > 0 \Rightarrow y^3 > 0$ (a ordem permanece intacta em cada produto).
	
	Finalmente pela lei da transitividade temos $x^3 < 0$ e $0 < y^3 \Rightarrow x^3 < y^3$.
	
	\item
		\begin{enumerate}
			\item 
			$x.0 = x(0)$, (A1)\\
			$ x(0) = x(z + (-z))$, (A4)\\ 
			$x(z + (-z)) = xz - xz $, (D)\\
			$xz - xz = xz + (-xz) $, (A1)\\ 
			Finalmente temos: $xz + (-xz) = 0$, (A4).\\
			\item
			Para o primeiro caso:\\
			$x + (-x) = 0$, (A4)\\							
			$y(x + (-x)) = y.0$, combinando (O2) com (OM)\\
			$yx + y(-x) = 0$, (D) e (a) acima\\
			$xy + (-x)y = 0$, (M2)\\						
			$xy + (-x)y + (-xy) = -xy$, combinando (O2) com (OA)\\
			$(-x)y + xy + (-xy) = -xy$, (A2)\\
			$(-x)y = -xy$, (A3)\\	
			\\	
			No segundo caso:\\
			$y + (-y) = 0$, (A4)\\							
			$x(y + (-y)) = x.0$, combinando (O2) com (OM)\\
			$xy + x(-y) = 0$, (D) e (a) acima\\
			$xy + x(-y) + (-xy) = -xy$, combinando (O2) com (OA)\\
			$x(-y) + xy + (-xy) = -xy$, (A2)\\
			$x(-y) = -xy$, (A3)\\		
			\\
			No terceiro:\\
			$(-x) + x = 0$, (A4) e (A2)\\
			$(-y)((-x) + x) = (-y).0$, combinando (O2) com (OM)\\
			$(-y)(-x) + (-y)x = 0$, (D) e (a) acima\\
			$(-x)(-y) + x(-y)$, (M2)\\
			$(-x)(-y) + x(-y) + xy = xy$, combinando (O2) com (OA)\\
			$(-x)(-y) = xy$, (A4)\\
			\item
			$x \leq 0$ ou $0 \leq x$, (O4)\\
			Se $x \leq 0$:\\
			$x - (-x) \leq 0 + (-x)$, (OA)\\
			$0 \leq -x$, (A4)\\
			$(-x)0 \leq (-x)(-x)$, (OM)\\
			Considerando o item (a) acima, temos:\\
			$0 \leq x^2$.\\
			Se $x \geq 0$:\\
			$xx \geq x.0$, (OM)\\
			Considerando o item (a) acima, temos:\\
			$x^2 \geq 0$.			
			\item
			$0 \leq 1$ e $0 \leq 1$ nos dá $0.1 \leq 1.1 = 1^2$, (OM)\\
			Por (M3) $1.1 = 1$ com $1 \neq 0$\\
			Logo temos $1^2 > 0$.
		\end{enumerate}
\end{enumerate}
\section{Módulo de um Número Real}
\begin{enumerate}
	\setcounter{enumi}{0}
	\item
	\begin{enumerate}
		\item
		$\vert-5\vert + \vert-2\vert = -(-5) - (-2) = 5 + 2 = 7$.
		\item
		$\vert-5 + 8\vert = \vert 3 \vert = 3$.
		\item
		$\vert-a\vert = -(-a) = a$.
	\end{enumerate}
	\item
	\begin{enumerate}
		\item
		$\vert x \vert = 2$\\
		$x = 2$ ou $x = - 2$.
		\item
		$\vert x + 1 \vert = 3$
		$x + 1 > 0 \Rightarrow x + 1 = 3 \Rightarrow x = 2$\\
		$x + 1 \leq 0 \Rightarrow -(x + 1) = 3 \Rightarrow x + 1 = -3 \Rightarrow x = -4$.
		\item
		$\vert 2x - 1 \vert = 1$\\
		$2x - 1 > 0 \Rightarrow 2x - 1 = 1 \Rightarrow 2x = 2 \Rightarrow x = 1$\\
		$2x - 1 \leq 0 \Rightarrow -(2x - 1) = 1 \Rightarrow 2x - 1 = -1 \Rightarrow 2x = 0 \Rightarrow x = 0$.		
	\end{enumerate}
	\item
	\begin{enumerate}
		\item
		$\vert x \vert \leq 1$\\
		$ x > 0 \Rightarrow x \leq 1$\\
		$ x \leq 0 \Rightarrow -x \leq 1 \Rightarrow x \geq -1$\\
		$-1 \leq x \leq 1$
		\item
		$\vert 2x - 1 \vert < 3$\\		
		$2x - 1 > 0 \Rightarrow 2x - 1 < 3 \Rightarrow 2x < 4 \Rightarrow x < 2$\\
		$2x - 1 < 0 \Rightarrow -(2x - 1) < 3 \Rightarrow 2x - 1 > -3  \Rightarrow 2x > -2 \Rightarrow x > -1$\\
		$-1 < x < 2$\\
		\item
		$\vert 2x - 1 \vert < -2$, não admite solução pois o módulo de um número real é sempre positivo ou igual à 0.
		\item
		$\vert 2x - 1 \vert < \frac{1}{3}$\\
		$-\frac{1}{3} < 2x - 1 < \frac{1}{3} \Rightarrow -\frac{1}{3} + 1 < 3x < \frac{1}{3} + 1 \Rightarrow$\\
		$\frac{2}{3} < 3x < \frac{4}{3} \Rightarrow \frac{2}{9} < x < \frac{4}{9}$.
		\item
		$\vert 2x^2 - 1 \vert < 1$\\
		$\vert 2x^2 - 1 \vert > 0 \Rightarrow 2x^2 - 1 < 1 \Rightarrow 2x^2 < 2 \Rightarrow x^2 < 1 \Rightarrow x < 1$ ou $x > -1$ com $x \neq 0$.
		$\vert 2x^2 - 1 \vert \leq 0 \Rightarrow 2x^2 - 1 > -1 \Rightarrow 2x^2 > 0 \Rightarrow x^2 > 0 \Rightarrow x \neq 0$\\
		$-1 < x < 1$, $x \neq 0$.
		\item
		$\vert x - 3 \vert < 4$\\
		$-4 <  x - 3 < 4 \Rightarrow -1 < x < 7$.
		\item
		$\vert x \vert > 3$\\		
		$x > 0 \Rightarrow x > 3$\\
		$x \leq 0 \Rightarrow -x > 3 \Rightarrow x < -3$\\
		$x < -3$ ou $x > 3$.
	\end{enumerate}
	\item 
		Dado $r > 0$, provar: \\
		$\vert x \vert > r \Leftrightarrow x < -r$ ou $x > r$\\\\
		$x > 0 \Rightarrow x > r$\\
		$x \leq 0 \Rightarrow -x > r \Rightarrow x < -r$\\
		Logo $\vert x \vert > r \Rightarrow x < -r$ ou $x > r$.\\
		Por outro lado:\\
		$x > r$ com $r > 0 \Rightarrow x^2 > r^2 \Rightarrow \sqrt{x^2} > \sqrt{r^2} \Rightarrow \vert x \vert > r$.\\
		No caso de $x < -r$ com $r > 0$ temos:\\
		$x < -r$ com $x < 0 \Rightarrow -x > r \Rightarrow (-x)^2 > r^2\Rightarrow \sqrt{(-x)^2} > \sqrt{r^2} \Rightarrow \vert x \vert > r$.\\
	\item
		\begin{enumerate}
			\item
			$\vert x + 1 \vert + \vert x \vert$\\
			Devemos averiguar as quatro combinações de sinais para as duas expressões nos módulos:\\
			Para $x + 1 > 0$ e $x > 0$, temos $x > -1$ e $x > 0$, ou seja, $x > 0$:\\
			$x + 1 + x = 2x + 1$\\
			Para $x + 1 > 0$ e $x \leq 0$, temos $x > -1$ e $x \leq 0$, ou seja, $-1 < x \leq 0$:\\
			$x + 1 - x = 1$\\
			Para $x + 1 \leq 0$ e $x > 0$, temos $x < -1$ e $x > 0$, que não é possível.\\
			Para $x + 1 \leq 0$ e $x \leq 0$, temos $x \leq -1$ e $x \leq 0$, ou seja, $x \leq -1$:\\		
			$-(x + 1) -x = -2x - 1$\\
			Logo a solução é:\\
			\begin{equation*}
		    	\vert x + 1 \vert + \vert x \vert =
			    \begin{cases}
			      -2x - 1, & \text{se}\ x \leq -1 \\
			      \hfill 1, & \text{se}\ -1 < x \leq 0 \\
			      \hfill 2x + 1, & \text{se}\ x > 0 
		    	\end{cases}
			\end{equation*}
			\item
			$\vert x - 2\vert - \vert x + 1 \vert$\\
			Para $x - 2 > 0$ e $x + 1 > 0$, temos $x > 2$ e $x > -1$, ou seja, $x > 2$:\\
			$x - 2 - x - 1 = -3$\\
			Para $x - 2 > 0$ e $x + 1 \leq 0$, não é possível haver $x > 2$ e $ x < -1$.\\
			Para $x - 2 \leq 0$ e $x + 1 > 0$, temos $x \leq 2$ e $x > -1$, ou seja, $-1 < x \leq 2$:\\
			$-(x - 2) - (x + 1) = -x + 2 - x - 1 = -2x + 1$\\
			Para $x - 2 \leq 0$ e $x + 1 \leq 0$, temos $x \leq 2$ e $x \leq -1$, ou seja, $x \leq -1$:\\
			$-(x - 2) - [-(x + 1)] = -x + 2 + x + 1 = 3$\\
			Logo a solução é:\\
			\begin{equation*}
		    	\vert x - 2 \vert - \vert x + 1 \vert =
			    \begin{cases}
			      \hfill 3, & \text{se}\ x \leq -1 \\
			      -2x + 1, & \text{se}\ -1 < x \leq 2 \\
			      \hfill -3, & \text{se}\ x > 2 
		    	\end{cases}
			\end{equation*}
		\end{enumerate}
\end{enumerate}
\end{document}
